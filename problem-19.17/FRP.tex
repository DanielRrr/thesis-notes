\documentclass[a4paper]{article}
\usepackage{amsmath}
\usepackage{amsthm}
\usepackage{amsfonts}
\usepackage{amssymb}
\usepackage{bussproofs}
\usepackage{mathtools}
\usepackage{verbatim}
\usepackage{dsfont}
\usepackage{mathabx}
\usepackage[all, 2cell]{xy}
\usepackage[all]{xy}
\usepackage{wasysym}
\usepackage{rotating}
\usepackage{geometry}
\usepackage{trfsigns}
\usepackage{cmll}
\usepackage{authblk}
\usepackage{hyperref}
\usepackage{cleveref}
\usepackage{lipsum}
\usepackage{extpfeil}
\usepackage{soul}
\usepackage{graphicx}

\newcommand\mapsfrom{\mathrel{\reflectbox{\ensuremath{\mapsto}}}}

\theoremstyle{defin}
\newtheorem{defin}{Definition}

\theoremstyle{theorem}
\newtheorem{theorem}{Theorem}

\theoremstyle{prop}
\newtheorem{prop}{Proposition}

\theoremstyle{lemma}
\newtheorem{lemma}{Lemma}

\theoremstyle{ex}
\newtheorem{ex}{Example}

\theoremstyle{col}
\newtheorem{col}{Corollary}

\theoremstyle{claim}
\newtheorem{claim}{Claim}

\author{Daniel Rogozin}
\date{}
\title{The finite base property for subreducts of representable relation algebras}

\begin{document}

\maketitle

\section{Relation algebras and their reducts}

We recall the basic definitions and results about relation algebras \cite{hirsch2002relation} \cite{maddux2006relation}.

\begin{defin}
  $ $

    A relation algebra is an algebra $\mathcal{R} = \langle R, 0, 1, +, -, ;, {}^{\smile}, {\bf 1 }\rangle$ such that $\langle R, 0, 1, +, - \rangle$ is a Boolean algebra and the following
    equations hold, for each $a, b, c \in R$:
    \begin{enumerate}
      \item $a ; (b ; c) = (a ; b) ; c$
      \item $(a + b) ; c = (a ; c) + (b ; c)$
      \item $a ; {\bf 1} = a$
      \item $a^{\smile \smile} = a$
      \item $(a + b)^{\smile} = a^{\smile} + b^{\smile}$
      \item $(a ; b)^{\smile} = b^{\smile} ; a^{\smile}$
      \item $a^{\smile} ; (- (a ; b)) \leq - b$
    \end{enumerate}
    where $a \leq b$ iff $a + b = b$. ${\bf RA}$ denotes the class of all relation
    algebras.
\end{defin}
A relation algebra is called symmetric, if every element is self-converse. A relation algebra is called integral, if
\begin{center}
  $a ; b = 0 \Rightarrow a = 0 \text{ or } b = 0$.
\end{center}

\begin{defin}
    A proper relation algebra is an algebra $\mathcal{R} = \langle R, 0, 1, \cup, -, |, {}^{\smile}, {\bf 1 }\rangle$ such that $R \subseteq \mathcal{P}(W)$, where $W$ is an equivalence relation; $0 = \emptyset$; $1 = W$; $\cap$, $\cup$, $-$ are set-theoretic intersection, union, and
    complement respectively; $|$ is relation composition, ${}^{\smile}$ is relation converse, ${\bf 1}$ is a
    diagonal relation restricted to $W$, that is:
    \begin{enumerate}
    \item $a | b = \{ \langle x, z \rangle \: | \: \exists y \: \langle x, y \rangle \in a \: \& \: \langle y, z \rangle \in b \}$
    \item $a^{\smile} = \{ \langle x, y \rangle \: | \: \langle y, x \rangle \in a \}$
    \item ${\bf 1} = \{ \langle x, y \rangle \: | \: x = y \}$
    \end{enumerate}
      The class of all proper relation algebras is denoted as ${\bf PRA}$. ${\bf Rs}$ is the class of all relation set algebras, proper relation algebra with a diagonal subrelation as an identity. ${\bf RRA}$ is the class of all representable relation algebras, that is, the closure of ${\bf PRA}$ under isomorphic copies. That is,
      ${\bf RRA} = {\bf IPRA}$.
\end{defin}
Note that the (quasi)equational theories of these classes coincide, that is
\begin{center}
  ${\bf IPRA} = {\bf RRA} = {\bf SP}{\bf Rs}$
\end{center}
It is also known that ${\bf RA}$ and ${\bf RRA}$ are varieties, but ${\bf RRA}$ is not finitely axiomatisable.

We consider signatures include the equality predicate and are subsets of
$\Omega = \{ =, \leq, 0, 1, {\bf 1}, +, \cdot, -, \smile, ;, \setminus, / \}$. A subset $\tau \subseteq \Omega$ is called a $\tau$-signature.

\begin{defin}
  An interpretation $R$ of $\tau$-algebra $\mathcal{A}$ over a base $X$ is a map that assigns a binary relation $a^R \subseteq X \times X$
  for each $a \in X$. An interpretation $R$ is a $\tau$-interpretation if the following hold
  (for the relevant predicate, function or constant belonging to $\tau$):
  \begin{enumerate}
    \item $a^R = b^R$ iff $a = b$,
    \item $a^R \subseteq b^R$ iff $a \leq b$
    \item $0^{R} = \emptyset$,
    \item $(a + b)^{R} = a^R \cup b^R$,
    \item $(a \cdot b)^{R} = a^R \cap b^R$,
    \item $(- a)^R = \{ (x, y) \: | \: \exists b \in \mathcal{A} \: (x, y) \in b^R \setminus a^R \}$,
    \item $({\bf 1}) = \{ (a, a) \: | \: a \in \mathcal{A} \}$,
    \item $(a;b)^R = \{ (x, y) \: | \: \exists z \in X \: (x, z) \in a^R \: \& \: (z, x) \in b^R \}$,
    \item $(a \setminus b)^{R} = \{ (x, y) \: | \: \forall z \in X \: ((z, x) \in a^R \Rightarrow (z, y) \in b^R)\}$,
    \item $(a / b)^{R} = \{ (x, y) \: | \: \forall z \in X \: ((y, z) \in a^R \Rightarrow (x, z) \in b^R)\}$.
  \end{enumerate}
\end{defin}

We will the following notations according to, for example, \cite{hirsch2011positive}. Let $\tau$ be a subsets of operations definable in ${\bf RRA}$. ${\bf R}(\tau)$ is the class of subalgebras of $\tau$-reducts of algebras belonging to ${\bf Rs}$.

A $\tau$-structure is \emph{representable} if its isomorphic to some algebra of relations of $\tau$-signature. A representable
finite $\tau$-structure has a \emph{finite representation over a finite base} if it is isomoprhic to some finite representable over a
finite base.
${\bf R}(\tau)$ has the finite representation property if every $\mathcal{A} \in {\bf R}(\tau)$ has a finite representation over a
finite base.

\subsection{Residuals}

One may express residuals in every $\mathcal{R} \in {\bf RA}$ as follows using Boolean negation, inversion, and composition as follows:

\begin{enumerate}
  \item $a \setminus b = -(a^{\smile} ; -b)$
  \item $a / b = - (- a ; b^{\smile})$
\end{enumerate}

Those residuals have the following interpretation in $\mathcal{R} \in {\bf PRA}$ (as well as in ${\bf RRA}$), for every $a, b \in \mathcal{R}$:
\begin{enumerate}
  \item $a \setminus b = \{ \langle x, y \rangle \: | \: \forall z \: (z, x) \in a \Rightarrow (z, y) \in b \}$
  \item $a / b = \{ \langle x, y \rangle \: | \: \forall z \: (y, z) \in b \Rightarrow (x, z) \in a \}$
\end{enumerate}
One may illustrate composition and residuals in ${\bf PRA}$ and ${\bf RRA}$ via the following triangles:

\xymatrix{
&&& \exists y \ar@{-->}[ddr]^{b} &&& \forall z \ar@{-->}[ddl]_{a} \ar@{-->}[ddr]^{b} &&& \forall z \\
&&&&&& \Rightarrow &&& \Leftarrow \\
&& x \ar@{-->}[uur]^{a} \ar[rr]_{a;c} && z & x \ar[rr]_{a \setminus b} && y & x \ar@{-->}[uur]^{a} \ar[rr]_{a / b} && y \ar@{-->}[uul]_{b}
}

\section{The ${\bf R}(;, \setminus, /, \leq)$ case}

 A \emph{relation semigroup} (${\bf RS}$) is a algebra of the signature $\tau = \langle ;, \setminus, /, \leq \rangle$, where $;, \setminus, /$ are binary function symbols and $\leq$ is a binary relation symbol.

\begin{defin}
  A residuated semigroup is an algebra $\mathcal{A} = \langle A, ;, \leq, \setminus, / \rangle$ such that $\langle A, ;, \leq, \rangle$ is
  an ordered residuated semigroup and the following equivalences hold for each $a, b, c \in A$:

  \begin{center}
    $b \leq a \setminus c \Leftrightarrow a ; b \leq c \Leftrightarrow a \leq c / b$
  \end{center}
  ${\bf ORS}$ is the class of all residuated semigroups.
\end{defin}

Historically, residuated structures were introduced by Krull to study ideals of rings \cite{krull1968idealtheorie}. Such algebras further has been considered within problems of semantics of substructural logics, see \cite{jipsen2002survey}.

Let us define the class ${\bf R}(;, \setminus, /, \leq)$ explicitly:

\begin{defin} \label{rrs}
  Let $A$ be a set of binary relations on some base set $W$ such that $R = \cup A$ is transitive and $\{ x, y \: | \: x R y \} = W$. A relation residuated semigroup is an algebra $\mathcal{A} = \langle A, ;, \setminus, /, \subseteq \rangle$ where for each $r, s \in A$
  \begin{enumerate}
    \item $r ; s = \{ \langle a, c \rangle \: | \: \exists b \in W \: (\langle a, b \rangle \in r \: \& \: \langle b, c \rangle \in s) \}$
    \item $r \setminus s = \{ \langle a, c \rangle \: | \: \forall b \in W \: (\langle b, a \rangle \in r \Rightarrow \langle b, c \rangle \in s)\}$
    \item $r / s = \{ \langle a, c \rangle \: | \: \forall b \in W \: (\langle c, b \rangle \in s \Rightarrow \langle a, b \rangle \in r)\}$
    \item $r \leq s$ iff $r \subseteq s$.
  \end{enumerate}
\end{defin}
Relation residuated semigroup are also called representable relativised relational structure (${\bf RRS}$).
Andr\'{e}ka and Mikul\'{a}s proved the following representation theorem for ${\bf ORS}$ in \cite{andreka1994lambek} that implies relational completeness of the Lambek calculus, the logic of ${\bf ORS}$:

\begin{theorem}\label{ors=irrs}
  ${\bf ORS} = {\bf IRRS}$, where ${\bf IRRS}$ is a closure of ${\bf RRS}$ under isomorphic copies.
\end{theorem}
One may reformulate this result as $\mathcal{A} \in {\bf R}(;, \setminus, /, \leq)$ iff $\mathcal{A}$ is a residuated semigroup.

\subsection{Quantales and quantic nuclei}

A quantic nucleus is a closure operator on an ordered semigroup allowing one to define subalgebras. Such an operator is a generalisation of a well-known nucleus on a Heyting algebra, see \cite{bezhanishvili2016locales}. The following definition and the proposition below are given according to \cite{galatos2007residuated} and \cite{rosenthal1990quantales}.
\begin{defin}
  A quantic nucleus on a partially ordered semigroup $\langle A, ;, \leq \rangle$ is a map $j : A \to A$ such that $j$ a closure operator such that $j a ; j b \leq j (a ; b)$.
\end{defin}

\begin{prop} \label{subsemi}
  Let $\mathcal{A} = \langle A, ;, \leq \rangle$ be a partially ordered semigroup and $j$ a quantic conucleus, then
  $\mathcal{A}_j = \{ a \in A \: | \: j a \leq a \}$ is a partially ordered subsemigroup, where $a ;_j b = j(a ; b)$.
\end{prop}

\begin{defin}
  A quantale is an algebra $\mathcal{Q} = \langle Q, \cdot, \bigvee \rangle$ such that $\mathcal{Q} = \langle Q, \bigvee \rangle$ is a complete lattice, $\langle Q, \cdot \rangle$ is a semigroup, and the following conditions hold for each $a \in Q$ and $A \subseteq Q$:
  \begin{enumerate}
    \item $a \cdot \bigvee A = \bigvee \{ a \cdot q \: | \: q \in A \}$,
    \item $\bigvee A \cdot a = \bigvee \{ q \cdot a \: | \: q \in A \}$.
  \end{enumerate}
\end{defin}
Note that any quantale is a residuated semigroup as well. One may express residuals uniquely with suprema and products as follows:
\begin{enumerate}
  \item $a \setminus b = \bigvee \{ c \: | \: a \cdot c \leq b \}$,
  \item $a / b = \bigvee \{ c \: | \: b \cdot c \leq a \}$.
\end{enumerate}

One may embed any residuated semigroup into a quantale via the Dedekind-MacNeille completion (see, for example, \cite{theunissen2007macneille}) as follows. According to Goldblatt \cite{goldblatt2006kripke}, residuated semigroups have the following representation that exploits quantic conuclei (a closure operator over ordered semigroup) and Galois connection:
\begin{theorem} \label{orsRep}
  Every residuated semigroup has an isomorpic embedding to the subalgebra of some quantale.
\end{theorem}
\begin{proof}
  We provide a proof sketch.

  Let $\mathcal{A} = \langle A, \leq, \cdot, \setminus, / \rangle$ be a residuated semigroup.

  Let $X \subseteq A$. We put $lX$ and $uX$ as the sets of lower and upper bounds of $X$ in $A$. We also put $m X = lu X$.
  Note that the lower cone of an arbitrary $x$, $\downarrow x = \{ y \: | \: S \ni y \leq x\}$, is $m$-closed, that is,
  $m (\downarrow x) = \downarrow x$. Moreover, $m : \mathcal{P}(A) \to \mathcal{P}(A)$ is a closure operator and
  $\langle (\mathcal{P}(A))_m, \subseteq \rangle$ (where $(\mathcal{P}(A))_m = \{ X \in \mathcal{P}(S) \: | \: m X = X\}$ ) is a complete
  lattice with $\bigvee_{m} \mathcal{X} = m ( \bigcup \mathcal{X})$ and $\bigwedge_{m} = \bigcap \mathcal{X}$ \cite{davey2002introduction}.
  Moreover, according to Proposition~\ref{subsemi}, $\langle (\mathcal{P}(A))_m, \subseteq, ;_m \rangle$ is a subquantale of
  $\langle (\mathcal{P}(S))_m, \subseteq, ; \rangle$, since $m$ is a quantic nucleus. Here $X ; Y = \{ x ; y \: | \: x \in X, y \in Y \}$ and $X ;_m Y = m (X ; Y)$.

  We define a map $f_m : \mathcal{A} \to (\mathcal{P}(A))_m$ such that $f_m : a \mapsto \downarrow a$. Note that $f_m$ preserves residuals.
\end{proof}

Note that this (more or less) construction holds for residuated monoids as well, see \cite{goldblatt2011grishin}.

In their turn, quantales have a relational representation. First of all, let us define a relational quantale.
The notion of relational quantale was introduced by Brown and Gurr to represent quantales as algebras of relations and study relational semantics of the full Lambek calculus, see \cite{brown1993representation} and \cite{brown1995relations}.
\begin{defin}
  Let $A$ be a set. A relational quantale on $A$ is an algebra $\langle R, \subseteq, ; \rangle$, where
  \begin{enumerate}
    \item $R \subseteq \mathcal{P}(A \times A)$,
    \item $\langle R, \subseteq \rangle$ is a complete join-semilattice,
    \item $;$ is a relational composition that respects suprema in each coordinate.
  \end{enumerate}
\end{defin}

The following representation theorem for quantale has been proved by Brown and Gurr \cite{brown1993representation}.
\begin{theorem} \label{quantaleRep}
  Every complete residuated semigroup $\mathcal{A}$ (quantale) is isomorphic to relational quantale on the underlying set of $\mathcal{A}$.
\end{theorem}

\begin{proof}
  Let us describe a proof sketch. Let $\mathcal{Q}$ be a quantale and $\mathcal{G}(\mathcal{Q})$ a set of its generators. We define:

  \begin{center}
    $\hat{a} = \{ \langle g, q \rangle \: | g \in \mathcal{G}(\mathcal{Q}), q \in \mathcal{Q}, g \leq a \cdot q \}$

    $\widehat{\mathcal{Q}} = \{ \hat{a} \: | \: a \in \mathcal{Q} \}$
  \end{center}

  The rest of the sketch consists of the following claims.

  {\bf Claim 1} $a \leq b$ iff $\hat{a} \subseteq \hat{b}$.

  {\bf Claim 2} $\widehat{\bigvee A} = \bigvee \widehat{A}$, $\hat{a} ; \hat{b} = \widehat{a \cdot b}$, and $\langle \widehat{\mathcal{Q}}, \subseteq \rangle$ is a complete semilattice.

  {\bf Claim 3}. $\langle \widehat{\mathcal{Q}}, \subseteq, ; \rangle$ is a relational quantale.

  {\bf Claim 4}. $\mathcal{Q}$ is isomorphic to $\langle \widehat{\mathcal{Q}}, \subseteq, ; \rangle$.
\end{proof}

Theorems \ref{orsRep} and \ref{quantaleRep} imply the following statement.
\begin{col} \label{orsRep2}
  Every residuated semigroup is isomorphic to some subalgebra of some relational quantale.
\end{col}

\subsection{Solution for Problem 19.17 \cite{hirsch2002relation}}

In \cite{hirsch2002relation}, Hirsch and Hoskidson left the problem about having the finite representation property. Given a finite algebra $\mathcal{A} \in {\bf RRS}$, we show that $\mathcal{A}$ is isomorphic to some algebra $\mathcal{B}$ that has the finite base.

First of all, let us show that
\begin{lemma} \label{rrsLemma}
  Let $\mathcal{A} = \langle A, \cdot, \setminus, /, \leq \rangle$ be a residuated semigroup, then its representation with relational quantales belongs to ${\bf RRS}$.
\end{lemma}

\begin{proof}
  By Corollary~\ref{orsRep2}, $\mathcal{A}$ is isomorphic subalgebra of $\widehat{\mathcal{Q}_{\mathcal{A}}}$, where
  $\widehat{\mathcal{Q}_{\mathcal{A}}} = \langle R, ;, \bigvee \rangle$ is a quantale of Theorem~\ref{orsRep}.
  $\mathcal{Q}_{\mathcal{A}}$ is the quantale of Galois stable subsets of $\mathcal{A}$.

  $\widehat{\mathcal{Q}_{\mathcal{A}}}$ is the relational quantale of binary relations on $\mathcal{Q}_{\mathcal{A}}$ as a base set and
  $R = \{ \hat{A} \: | \: A \in \mathcal{Q}_{\mathcal{A}}\}$.

  Let us show that $W = \bigcup R$ is a transitive relation.
  Suppose $\widehat{X}, \widehat{Y}, \widehat{Z} \in R$ for $X, Y, Z \in \mathcal{Q}_{\mathcal{A}}$ with
  $\widehat{X} W \widehat{Y} W \widehat{Z}$.

  $\widehat{X} W \widehat{Y}$ means that $(\widehat{X}, \widehat{Y}) \in W$, i.e., $(\widehat{X}, \widehat{Y}) \in R_1$ for some
  $R_1 \in R$.

  $\widehat{Y} W \widehat{Z}$ means that $(\widehat{Y}, \widehat{Z}) \in W$, i.e., $(\widehat{Y}, \widehat{Z}) \in R_2$ for some
  $R_2 \in R$.

  We need $(\widehat{X}, \widehat{Z}) \in W$, that is, there exists $R_3 \in W$ such that $(\widehat{X}, \widehat{Z}) \in R_3$.
  But $(\widehat{X}, \widehat{Z}) \in R_1 ; R_2$. But $m(R_1 ; R_2) \in R$ as well, thus, $(\widehat{X}, \widehat{Z}) \in W$.
  We put $R_3 = m(R_1 ; R_2)$, where $m$ is the associated quantic nucleus on $\mathcal{P}(A)$.

  Thus, a representation of $\mathcal{A}$ by means of Theorem~\ref{orsRep} is representable in terms of ``usual'' representations.
  Thus, ${\bf RRS}$ is closed under such representations.
\end{proof}

\begin{col}\label{finite}
  Let $\mathcal{A}$ be a finite residuated semigroup, then its representation with the corresponding relational quantale has the finite base.
\end{col}

\begin{proof}
  The base of $\widehat{\mathcal{Q}_{\mathcal{A}}}$ is the set of Galois stable subsets of $\mathcal{A}$, the cardinality of which is clearly finite.
\end{proof}


Theorem \ref{orsRep}, Theorem \ref{quantaleRep}, Lemma \ref{rrsLemma}, and Corollary \ref{finite} provide the solution for Problem 19.17 of \cite{hirsch2002relation}.
\begin{theorem} \label{solution}
  Let $\mathcal{A} \in {\bf RRS}$ and $|\mathcal{A}| < \omega$, then there exists a set $W$ such that $|W| < \omega$, a set $A$ of binary relations on $W$, $R = \cup A$ with $dom(R) = W$ such that
  $\mathcal{A} \cong \langle A, ;, \setminus, /, \subseteq \rangle$.
\end{theorem}
Clearly that a desired $W$ is the domain of $\mathcal{A}$.

The main corollary of Theorem~\ref{solution} is that the Lambek calculus has the finite model property. Thus, we have a semantical proof of decidability of the Lambek calculus. Before that, there were several algebraic proofs that the Lambek caclulus has the fmp \cite{buszkowski2008infinitary}, but the authors considered arbitrary algebras, not representable ones.
Alternatively, one may show that the Lambek calculus is decidable syntactically, that is, via cut elimination and the subformula property \cite{lambek1958mathematics}.
\begin{col}
  The Lambek calculus is complete w.r.t finite relational models (has the fmp).
\end{col}

Modulo the famous result by Pentus \cite{pentus2006lambek}, one may also conclude that

\begin{col}
  The theory of ${\bf IRRS}$ is $\operatorname{NP}$-complete.
\end{col}

\subsection{The second proof for Problem 19.17 (the case of residuated monoids)}

The construction above does not work properly for residuated monoids since the identity has the representation $\hat{{\bf 1}} = \{ (a,b) \: | \: a \leq {\bf 1} ; b, a \in \mathcal{G}(\mathcal{Q}), b \in \mathcal{Q}\} = \{ (a,b) \: | \: a \leq b, a \in \mathcal{G}(\mathcal{Q}), b \in \mathcal{Q}\}$, which is not the identity relation at all. Thus, the finite representation property for residuated monoids has to be proved differently.

First of all, let us recall we can prove the finite representation property might be shown for (ordered) monoids.

Let us put $\tau = \{ ;, {\bf 1} \}$. Recall that $\mathcal{A} \in {\bf R}(\tau)$ iff $\mathcal{A} = \langle A, ;, {\bf 1} \rangle$ is a monoid. That is, every monoid is representable. The finite representation property for finite monoids might be shown with a Cayley representation putting $A$ as a base set and defining:
\begin{center}
  $a^R = \{ (b, b ; a) \: | \: b \in A \}$
\end{center}
Thus, we have
\begin{lemma}
  Let $\tau \subseteq \{ ;,  \}$ ${\bf R}(\tau)$ has the finite representation property.
\end{lemma}

We also note that $\mathcal{A} \in {\bf R}(\leq)$ iff $\mathcal{A}$ is a poset defining a representation over the set of upper cones
$\operatorname{Up}(\mathcal{A})$ as
\begin{center}
  $a^R = \{ (X,X) \: | \: a \in X \in \operatorname{Up}(\mathcal{A}) \}$
\end{center}

In contrast to usual monoids, the universal theory of representable partially ordered monoids is not finitely axiomatisable \cite{hirsch2005class}.

\begin{lemma}
  ${\bf R}(;, {\bf 1}, \leq)$ has (?) the finite representation property.
\end{lemma}

\begin{proof}

\end{proof}

\begin{theorem}
  ${\bf R}(;, \setminus, /, {\bf 1}, \leq)$ has (?) the finite representation property.
\end{theorem}

\begin{proof}
\end{proof}

\section{Signatures containing either Boolean meet or negation}

It is known that the class ${\bf R}(\tau)$ has no the finite representation property where $\tau$ is either $\{ ;, \cdot, \smile\}$ or $\{ ;, \cdot, {\bf 1} \}$ \cite{hirsch2004finite}. That is, they have representations only over infinite bases. This fact has the following proof (a proof sketch, to be more precise).
\begin{proof}
Consider the set of atoms $\operatorname{A}(\mathcal{Q}) = \{ {\bf 1}, <, > \}$, where ${\bf 1}$ is the identity, $<^{\smile} = >$ and vice versa. The composition is defined with the following equations:
\begin{enumerate}
  \item $< ; < = <$,
  \item $< ; > = {\bf 1} + < + >$
\end{enumerate}

Let $R$ be a $\tau$-representation of $\mathcal{Q} \upharpoonright_{\tau}$ over the base $X$. $\leq$ is definable and $0$ is below $<$. Note that $0^{R} \leq <^R$, but it is not necessary that $0^R = \emptyset$. $< \neq 0$, thus, $(x,y) \in <^R \setminus 0^R$. If $(u, v) \in <^R \setminus 0^R$, then $u \neq v$.
The reason for $u \neq v$ is the following one. Suppose ${\bf 1} \in \tau$, then $(u,u) \in {\bf 1}^R$, so $u = v$ implies $(u, v) \in <^R \cdot {\bf 1} = 0^R$. Suppose $\smile \in \tau$,
then $u = v$ implies $(u,v) \in (< \cdot >)^R = 0$. Since $< \: \leq \: < ; <$ and $(u, v) \in <^R \setminus 0^R$, then there exists $w \in X$ such that $(u, w) \in <^R$ and $(u, w) \in <^R$. Hence, $(u, v), (v, W) \in <^R \setminus 0^R$. Thus, $<^R \setminus 0^R$ is reflexive and dense, but it is not transitive.

For each $n < \omega$ there exist $x_0, \dots, x_{n - 1} \in X$ such that for each $i < j < n$ one has $(x_i, x_j) \in <^R \setminus 0^R$. Thus, $n \leq |X|$ for each $n < \omega$.
\end{proof}

Moreover, the stronger theorem holds that has been proved by Maddux \cite{maddux2016finite}:
\begin{theorem}
  There exists a finite algebra $\mathcal{A} \in {\bf R}(\tau)$, where $\tau \subseteq \{ ;, \cdot \}$, such that $\mathcal{A}$ has no finite representation.
\end{theorem}

\begin{theorem}
  There exists a finite algebra $\mathcal{A} \in {\bf R}(\tau)$, where $\{ ;, \neg \} \subseteq$, such that $\mathcal{A}$ has no finite representation.
\end{theorem}

\begin{proof}
  Let $\operatorname{At}(\mathcal{A}) = \{ {\bf 1}, p, q \}$ be the set of atoms of a finite relation algebra $\mathcal{A}$ having the cardinality $8$, where ${\bf 1}$ is the identity.

  Let $\mathcal{A} \upharpoonright_{\tau}$ be the $\tau$-subreduct of $\mathcal{A}$ and $R$ its representation over the base $U$.

  {\bf Claim} For each $n < \omega$ there exists $x_0, \dots, x_{n - 1}$ such that for each $i < j < n$, $(x_i, x_j) \in X$, thus $|X| \geq \omega$.
\end{proof}

\begin{col}
  If $\tau$ contains composition and either Boolean meet or complement, then ${\bf R}(\tau)$ fails to have the finite representation property.
\end{col}

\begin{theorem}
  Suppose ${\bf R}(\tau)$ fails to have the finite representation property, then $\tau$ contains composition either meet or negation.
\end{theorem}

\bibliographystyle{plain}
\bibliography{Text}



\end{document}

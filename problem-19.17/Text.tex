\documentclass[a4paper]{article}
\usepackage{amsmath}
\usepackage{amsthm}
\usepackage{amsfonts}
\usepackage{amssymb}
\usepackage{bussproofs}
\usepackage{mathtools}
\usepackage{verbatim}
\usepackage{dsfont}
\usepackage{mathabx}
\usepackage[all, 2cell]{xy}
\usepackage[all]{xy}
\usepackage{wasysym}
\usepackage{rotating}
\usepackage{geometry}
\usepackage{trfsigns}
\usepackage{cmll}
\usepackage{authblk}
\usepackage{hyperref}
\usepackage{cleveref}
\usepackage{lipsum}
\usepackage{extpfeil}
\usepackage{soul}
\usepackage{graphicx}

\newcommand\mapsfrom{\mathrel{\reflectbox{\ensuremath{\mapsto}}}}

\theoremstyle{defin}
\newtheorem{defin}{Definition}

\theoremstyle{theorem}
\newtheorem{theorem}{Theorem}

\theoremstyle{prop}
\newtheorem{prop}{Proposition}

\theoremstyle{lemma}
\newtheorem{lemma}{Lemma}

\theoremstyle{ex}
\newtheorem{ex}{Example}


\theoremstyle{col}
\newtheorem{col}{Corollary}

\author{Daniel Rogozin}
\date{}
\title{The finite base property for some relation algebras subreducts}

\begin{document}

\maketitle

\nocite{*}

\section{The Relation Algebras Background}

We describe the basic definitions and results about relation algebras \cite{hirsch2002relation} \cite{maddux2006relation}.

\begin{defin}
  $ $

  \begin{enumerate}
    \item A relation algebra is an algebra $\mathcal{R} = \langle R, 0, 1, \wedge, \vee, \neg, ;, {}^{\smile}, {\bf 1 }\rangle$ such that $\langle R, 0, 1, \wedge, \vee, \neg \rangle$ is a Boolean algebra and the following
    equations hold, for each $a, b, c \in R$:
    \begin{enumerate}
      \item $a ; (b ; c) = (a ; b) ; c$
      \item $(a \vee b) ; c = (a ; c) \vee (b ; c)$
      \item $a ; {\bf 1} = a$
      \item $a^{\smile \smile} = a$
      \item $(a \vee b)^{\smile} = a^{\smile} \vee b^{\smile}$
      \item $(a ; b)^{\smile} = b^{\smile} ; a^{\smile}$
      \item $a^{\smile} ; (\neg (a ; b)) \leq \neg b$
    \end{enumerate}
    where $a \leq b$ iff $a \wedge b = a$ iff $a \vee b = b$. ${\bf RA}$ denotes the class of all relation
    algebras.
    \item A proper relation algebra is an algebra $\mathcal{R} = \langle R, 0, 1, \wedge, \vee, \neg, ;, {}^{\smile}, {\bf 1 }\rangle$ such that $R \subseteq \mathcal{P}(W)$, where $W$ is an equivalence relation; $0 = \emptyset$; $1 = W$; $\wedge$, $\vee$, $\neg$ are set-theoretic intersection, union, and
    complement respectively; $;$ is relation composition, ${}^{\smile}$ is relation converse, ${\bf 1}$ is a
    diagonal relation restricted to $W$, that is:
    \begin{enumerate}
    \item $a ; b = \{ \langle x, z \rangle \: | \: \exists y \: \langle x, y \rangle \in a \: \& \: \langle y, z \rangle \in b \}$
    \item $a^{\smile} = \{ \langle x, y \rangle \: | \: \langle y, x \rangle \in a \}$
    \item ${\bf 1} = \{ \langle x, y \rangle \: | \: x = y \}$
    \end{enumerate}
      The class of all proper relation algebras is denoted as ${\bf PRA}$. ${\bf Rs}$ is the class of all relation set algebras, proper relation algebra with a diagonal subrelation as an identity. ${\bf RRA}$ is the class of all representable relation algebras, that is, the closure of ${\bf PRA}$ under isomorphic copies. That is,
      ${\bf RRA} = {\bf IPRA}$.
  \end{enumerate}
\end{defin}
Note that the (quasi)equational theories of those classes coincide, that is
\begin{center}
  ${\bf IPRA} = {\bf RRA} = {\bf SP}{\bf Rs}$
\end{center}
Moreover, ${\bf RRA}$ is a variety, but it cannot be defined by any set of first-order formulas \cite{monk1964representable} \cite{}.

One may express residuals in every $\mathcal{R} \in {\bf RA}$ as follows, for every $a, b \in \mathcal{R}$:

\begin{enumerate}
  \item $a \setminus b = \neg (a^{\smile} ; \neg b)$
  \item $a / b = \neg (\neg a ; b^{\smile})$
\end{enumerate}

Those residuals have the following interpretation in $\mathcal{R} \in {\bf PRA}$ (as well as in ${\bf RRA}$), for every $a, b \in \mathcal{R}$:
\begin{enumerate}
  \item $a \setminus b = \{ \langle x, y \rangle \: | \: \forall z \: (z, x) \in a \Rightarrow (z, y) \in b \}$
  \item $a / b = \{ \langle x, y \rangle \: | \: \forall z \: (y, z) \in b \Rightarrow (x, z) \in a \}$
\end{enumerate}
One may illustrate composition and residuals in ${\bf PRA}$ and ${\bf RRA}$ via the following triangles:

\xymatrix{
&&& \exists y \ar@{-->}[ddr]^{b} &&& \forall z \ar@{-->}[ddl]_{a} \ar@{-->}[ddr]^{b} &&& \forall z \\
&&&&&& \Rightarrow &&& \Leftarrow \\
&& x \ar@{-->}[uur]^{a} \ar[rr]_{a;c} && z & x \ar[rr]_{a \setminus b} && y & x \ar@{-->}[uur]^{a} \ar[rr]_{a / b} && y \ar@{-->}[uul]_{b}
}

Given a subset of definable operations in ${\bf RA}$ $\tau$, we denote the class of subalgebras of the
$\tau$-reducts by ${\bf R}(\tau)$. The algebras containing to this class are defined as restrictions of
elements belonging to ${\bf Rs}$ to operations of $\tau$. By ${\bf Q}(\tau)$ we mean a quasivariety generated by $R(\tau)$. As in \cite{hirsch2011positive}, we put ${\bf Q}(\tau)$ as the closure of ${\bf R}(\tau)$ under subalgebras and products assuming that ${\bf R}(\tau)$ is already closed under ultraproducts.

\section{The Finite Base Property}

We recall the underlying definitions according to \cite[Section 19]{hirsch2002relation}

\begin{defin}
  Let ${\bf K}$ be a class of algebras of a signature $\Omega$, ${\bf K}$ has the finite algebra property, if
  if any first-order $\Omega$-sentence that is true in all finite algebras in ${\bf K}$ is true in every algebra in ${\bf K}$.
\end{defin}

The finite base property is a version of the finite algebra property if ${\bf K}$ is a class of representable algebras:

\begin{defin}
Let ${\bf K}$ be a class of representable algebras of a signature $\Omega$

  \begin{enumerate}
  \item ${\bf K}$ has the finite base property if any first-order $\Omega$-sentence that is true in every algebra in ${\bf K}$ having a representation over a finite base set is valid in ${\bf K}$.
  \item ${\bf K}$ has the finite algebra on finite base property if any finite algebra in ${\bf K}$ has a representation with finite base.
  \item ${\bf K}$ has the finite algebra property for equations/quasi-identites if any equation/quasi-identity that is true in all finite algebras is true in every algebra in ${\bf K}$. The finite base property for equations/quasi-identites is defined similarly.
\end{enumerate}
\end{defin}

The following statements were shown in \cite{andreka1999finite}. This lemma connects finite base property with finite algebra on finite base and finite algebra properties as follows:
\begin{lemma}
  Let ${\bf K}$ be a class of representable $\Omega$-algebras:
  \begin{enumerate}
    \item If ${\bf K}$ has the finite algebra property, then it has the finite algebra and the finite base properties for equations/quasi-identites.
    \item The finite algebra on finite base and the finite algebra properties implies the finite base property for ${\bf K}$. The same holds for equations/quasi-identites.
    \item If any representation of an infinite algebra has an infinite base, then the finite base property implies the finite algebra one for ${\bf K}$.
    \item Suppose $\Omega$ is finite and any subalgebra of a representable algebra is representable on the same base. Then the finite base property implies the finite algebra on finite base property.
  \end{enumerate}
\end{lemma}

\section{The Relation Residuated Semigroups Background}

\subsection{The underlying definitions and results}

 A \emph{relation structure} (${\bf RS}$) is an arbitrary algebra of the signature $\Omega = \langle \cdot, \setminus, /, \leq \rangle$, where $\cdot, \setminus, /$ are binary function symbols and $\leq$ is a binary relation symbol.

\begin{defin}
  A residuated semigroup is an algebra $\mathcal{S} = \langle S, \cdot, \leq, \setminus, / \rangle$ such that $\langle S, \cdot, \leq, \rangle$ is an ordered residuated semigroup and the following equivalences hold for each $a, b, c \in S$:

  \begin{center}
    $b \leq a \setminus c \Leftrightarrow a \cdot b \leq c \Leftrightarrow a \leq c / b$
  \end{center}
  ${\bf ORS}$ is the class of all residuated semigroups.
\end{defin}

\begin{defin} \label{rrs}
  Let $A$ be a set of binary relations on some base set $W$ such that $R = \cup A$ is transitive and $\{ x, y \: | \: x R y \} = W$. A relation residuated semigroup is an algebra $\mathcal{A} = \langle A, ;, \setminus, /, \subseteq \rangle$ where for each $r, s \in A$
  \begin{enumerate}
    \item $r ; s = \{ \langle a, c \rangle \: | \: \exists b \in W \: (\langle a, b \rangle \in r \: \& \: \langle b, c \rangle \in s) \}$
    \item $r \setminus s = \{ \langle a, c \rangle \: | \: \forall b \in W \: (\langle b, a \rangle \in r \Rightarrow \langle b, c \rangle \in s)\}$
    \item $r / s = \{ \langle a, c \rangle \: | \: \forall b \in W \: (\langle c, b \rangle \in s \Rightarrow \langle a, b \rangle \in r)\}$
  \end{enumerate}
\end{defin}
Relation residuated semigroup are also called representable relativised relational structure (${\bf RRS}$).

Andr\'{e}ka and Mikul\'{a}s proved the following representation theorem for ${\bf ORS}$ in \cite{andreka1994lambek} that implies relational completeness of the Lambek calculus, the logic of ${\bf ORS}$:

\begin{theorem}\label{ors=irrs}
  ${\bf ORS} = {\bf IRRS}$, where ${\bf IRRS}$ is a closure of ${\bf RRS}$ under isomorphic copies.
\end{theorem}

\subsection{The finite base property for ${\bf RRS}$}

\begin{defin}
  A relativised representation
\end{defin}

\begin{defin}
  The standard translation
\end{defin}

TODO: take a look at relativised representations and loosely guarded fragments in general
TODO: realise whether it makes sense to use the technique similar to \cite[Theorem 19.13]{hirsch2002relation}
used for weakly associative algebras.

\begin{theorem}
  Let $\mathcal{A}$ be a finite residuated semigroup and
  $|\mathcal{A}| < \omega$, then $\mathcal{A}$ has a finite relativised representation.
\end{theorem}

\begin{theorem}
  Let $\mathcal{A}$ be a finite representable residuated semigroup, then $\mathcal{A}$ is isomorphic to
  representable residuated semigroup a domain of which is finite.
\end{theorem}

\begin{proof}
  That might follow from the previous theorem, Theorem~\ref{ors=irrs}, and something else.
\end{proof}

\begin{col}
  The Lambek calculus has the fmp and the universal theory of ${\bf IRRS}$ is NP-complete.
\end{col}

The hypothetical plan is the following one:
\begin{enumerate}
\item Define properly relativised representation for residuated semigroups, that should look like ternary Kripke frames for the basic Lambek calculus or arrow logic.
\item Define the standard translation to such first-order relation structures. TODO: take a look at loosely guarded fragment stuff.
\item Every finite residuated semigroup has a finite relativised representation.
\item If every $\Pi_1$-statement $\varphi$ of the language of residuated semigroups that is valid in every
residuated semingroup is valid in algebra having a finite relativised representation (one may use here Theorem~\ref{ors=irrs} somehow), then $\varphi$ is valid in ${\bf ORS}$ as well as in ${\bf IRRS}$.
\item Every finite residuated semigroup should have a finite relativised representation.
\item Construct a finitely based relation residuated semigroup from that (an analogue of complex algebra or smth like that). This item is the most non-trivial one.
\item As a corollary, the first-order universal first-order theory of ${\bf IRRS}$ should be decidable and
(it seems so) $\operatorname{NP}$-complete (that should follow from the results in \cite{pentus2006lambek}). The Lambek calculus is decidable that was shown syntactically via cut elimination and subformula property.
Here we would have an alternative way of showing decidability for some substructural logics.
\end{enumerate}

\section{Join-semilattice ordered semigroups}

\begin{defin} A join-semilattice ordered semigroup (${\bf OS^{\vee}}$) is an algebra $\mathcal{S} = \langle S, \cdot, \vee \rangle$ such that $\langle S, \cdot \rangle$ is a semigroup, $\langle S, \vee \rangle$ is a join-semilattice and the following equations hold for each $a, b. c \in S$:

  \begin{enumerate}
    \item $a \cdot (b \vee c) = (a \cdot b) \vee (a \cdot c)$
    \item $(a \vee b) \cdot c = (a \cdot c) \vee (b \cdot c)$
  \end{enumerate}
\end{defin}
This class is clearly a variety since ${\bf OS^{\vee}}$ has the equational definition so far as $\vee$ is defined as an associative, idempotent, and commutative operation.

Let $A$ be a set of binary relations on some base set $W$ such that $R = \cup A$ is transitive and $\{ x, y \: | \: x R y \} = W$ as in Definition~\ref{rrs}. A relation join-semilattice ordered semigroup (${\bf ROS^{\vee}}$) is an algebra of binary relations $\mathcal{A} = \langle A, |, \cup \rangle$ such that $;$ is a relation composition as above and $\cup$ is the set-theoretic union.

Recall that a class of structures ${\bf K}$ is called finitely axiomatisable iff both ${\bf K}$ and its complement are closed
ultraproducts and isomorphic copies.

It is known that the class of all representable join-semilattice ordered semigroups has no finite axiomatisation \cite{andreka1989union}. In other words,

\begin{theorem}
  The equational and quasiequational theories of $R(;,\vee)$ is not finitely based.
\end{theorem}

Let us provide a proof of this fact using the rainbow technique \cite{hirsch2002relation} to show that the complement of ${\bf ROS^{\vee}}$ is not closed ultraproducts. This is (more or less) a standard way, see \cite{hodkinson2000axiomatizability}.
We note that representability is not decidable for finite relation algebras \cite{hirsch2001representability}. Moreover, representability is undecidable for lattice-ordered semigroups and ordered complemented semigroups \cite{neuzerling2016undecidability}.

First of all, we recall several definitions such as colourings. We define a sequence of relation algebras $\{ \mathfrak{A}_n \}_{n < \omega}$ each of which belongs to ${\bf RA}$. We need these algebras to show that their $\{;, \vee\}$-reducts are not representable. That is, we are seeking to show that

Given $n < \omega$, the set of atoms $\operatorname{At}(\mathfrak{A}_n)$ consists of the following elements:
\begin{itemize}
\item identity: ${\bf 1}$, an atom with no colour
\item greens: ${\bf g}_i$ for $0 \leq i \leq 2^n$
\item yellow: ${\bf y}$
\item black: ${\bf b}$
\item whites: ${\bf b}_{ij}$ for $0 \leq i \leq j \leq 2^n$
\item reds: ${\bf r}_i$ for $0 <i \leq 2^n$
\end{itemize}
We claim that every atom is self-converse ($a^{\smile} = a$). Given $x, y, z \in \mathfrak{A}_n$, a triple $(x,y,z)$ is an  inconsistent triangle if
\begin{center}
$x \land (y ; z) = y \land (z ; x) = z \land (x ; y) = 0$
\end{center}
We define the set of inconsistent triangles explicitly as follows.
\begin{itemize}
  \item $({\bf g}_i, {\bf g}_i, {\bf g}_i)$ for $0 \leq i \leq 2^n$
  \item $({\bf y}, {\bf y}, {\bf y})$ for $0 \leq i \leq 2^n$
  \item $({\bf g}_i, {\bf g}_i, {\bf w}_{kj})$ for $0 \leq i \leq 2^n$ and $0 \leq k \leq j \leq 2^n$
  \item $({\bf r}_i, {\bf r}_j, {\bf r}_k)$ unless $i + k = j$ or $i + k = j$ or $j + k = i$
  \item $({\bf g}_i, {\bf g}_{i + 1}, {\bf r})$ unless $j = 1$
  \item $({\bf g}_i, {\bf y}, {\bf w}_{jk})$ unless $i \in \{ j, k \}$
\end{itemize}
$({\bf g}_i, {\bf g}_i, {\bf w}_{kj})$ stands for ${\bf g}_i \land ({\bf g}_i ; {\bf w}_{kj}) = {\bf g}_i \land (z ; {\bf g}_i) = {\bf w}_{kj} \land ({\bf g}_i ; {\bf g}_i) = 0$, and so on.

\begin{lemma}
  For each $n < \omega$, $\mathfrak{A}_n$ does not belong ${\bf RRA}$. The $(\vee, ;)$-reduct $\mathfrak{S}_n$ of $\mathfrak{A}_n$ is not representable as well. For each $n < \omega$, there is an equation valid in set algebras failing in $\mathfrak{S}_n$.
\end{lemma}

\begin{proof}
  See \cite{hodkinson2000axiomatizability} to have a proof that $\mathfrak{A}_n \notin {\bf RRA}$.

  We prove that $\mathfrak{S}_n$ is not representable by contradiction.
  Suppose $h$ is an isomorphism of $\mathfrak{S}_n$ to a set relation of relations having similarity type $\{ ;, \vee \}$. Let $0$ be a zero element of $\mathfrak{A}_n$.
\end{proof}

TODO: define games and networks. Take a look at \cite{hirsch1997step}.

\begin{lemma}
  Any non-trivial ultraproduct of $\{ \mathfrak{A}_n \}_{n < \omega}$ is representable. The same statement for non-trivial ultraproduct of reducts $\{ \mathfrak{S}_n \}_{n < \omega}$.
\end{lemma}

\begin{lemma}
  TODO: one needs to realise when $\exists$ has a winning strategy
\end{lemma}

\subsection{The finite algebra on finite base for ${\bf RJSOS}$ (or its failure)}

\bibliographystyle{plain}
\bibliography{Text}

\end{document}

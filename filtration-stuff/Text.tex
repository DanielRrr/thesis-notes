\documentclass[a4paper]{article}
\usepackage{amsmath}
\usepackage{amsthm}
\usepackage{amsfonts}
\usepackage{amssymb}
\usepackage{bussproofs}
\usepackage{mathtools}
\usepackage{verbatim}
\usepackage{dsfont}
\usepackage{mathabx}
\usepackage[all, 2cell]{xy}
\usepackage[all]{xy}
\usepackage{wasysym}
\usepackage{rotating}
\usepackage{geometry}
\usepackage{trfsigns}
\usepackage{cmll}
\usepackage{authblk}
\usepackage{hyperref}
\usepackage{cleveref}
\usepackage{lipsum}
\usepackage{extpfeil}
\usepackage{soul}
\usepackage{graphicx}

\newcommand\mapsfrom{\mathrel{\reflectbox{\ensuremath{\mapsto}}}}

\theoremstyle{defin}
\newtheorem{defin}{Definition}

\theoremstyle{theorem}
\newtheorem{theorem}{Theorem}

\theoremstyle{prop}
\newtheorem{prop}{Proposition}

\theoremstyle{lemma}
\newtheorem{lemma}{Lemma}

\theoremstyle{ex}
\newtheorem{ex}{Example}


\theoremstyle{col}
\newtheorem{col}{Corollary}

\author{Daniel Rogozin}
\date{}
\title{Notes on filtration of logics containing {\bf K5}}

\begin{document}

\maketitle

\section{Preliminaries}

\begin{defin}
  An $n$-normal modal logic is a set of formulas that contains all Boolean tautologies, formulas $\Diamond_i p \lor \Diamond_i q \leftrightarrow \Diamond_i (p \lor q)$ and $\Diamond_i \bot \leftrightarrow \bot$ for $i \leq n$, and is closed under modus ponens, substitution, and monotonicity: from $\varphi \rightarrow \psi$ infer $\Diamond_i \varphi \rightarrow \Diamond_i \psi$ for $i \leq n$.
\end{defin}

\begin{defin} An $n$-Kripke model is a triple $\mathcal{M} = \langle W, R_1, \dots, R_n, \vartheta \rangle$, where $R_i \subseteq W \times W$, $\vartheta : \operatorname{PV} \to 2^W$, and the connectives have the following semantics:

  \begin{enumerate}
    \item $\mathcal{M}, w \models p \Leftrightarrow w \in \vartheta(p)$
    \item $\mathcal{M}, w \models \not \varphi \Leftrightarrow \mathcal{M}, w \nvDash \varphi$
    \item $\mathcal{M}, w \models \varphi \lor \psi \Leftrightarrow \mathcal{M}, w \models \varphi \text{ or } \mathcal{M}, w \models \psi$
    \item $\mathcal{M}, w \models \Diamond_i \varphi \Leftrightarrow \exists v \in R_i(w) \: \mathcal{M}, v \models \varphi$
  \end{enumerate}
\end{defin}

By ${\bf K5}$ we mean the logic ${\bf K} \oplus {A5}$, where ${A5} = \Diamond p \to \Box \Diamond p$. It is known that ${\bf K5}$ is the modal logic of all Euclidean frames. A frame is called Euclidean if for each $x, y, z$, $x R y$ and $x R z$ implies $y R z$.

\begin{prop} \label{K5proves}
  ${\bf K5}$ proves
  \begin{enumerate}
    \item $\Box^3 p \leftrightarrow \Box^2 p$
    \item $\Box^2 \Diamond p \leftrightarrow \Box \Diamond p$
    \item $\Box \Diamond \Box p \leftrightarrow \Box \Box p$
    \item $\Box \Diamond^2 p \leftrightarrow \Box \Diamond p$
  \end{enumerate}
\end{prop}

\begin{prop} \label{k5useful}
  Let $\mathcal{M}$ be a ${\bf K}5$ model, $x R y$ for $x, y \in W$ then one has
  \begin{center}
    $\mathcal{M}, x \models \Diamond \Box \varphi$ iff $\mathcal{M}, y \models \Diamond \Box \varphi$.
  \end{center}
\end{prop}

\begin{proof}
$ $

  \begin{enumerate}
    \item Suppose $\mathcal{M}, x \models \Diamond \Box \varphi$. One also has $\mathcal{M}, x \models \Diamond \Box \varphi \to \Box \Diamond \Box \varphi$, so $\mathcal{M}, x \models \Box \Diamond \Box \varphi$. Thus, $\mathcal{M}, y \models \Diamond \Box \varphi$ since $y \in R(x)$.
    \item Suppose $\mathcal{M}, y \models \Diamond \Box \varphi$, then $\mathcal{M}, y \models \Box \varphi$, so $\mathcal{M}, x \models \Diamond \Box \varphi$.
  \end{enumerate}
\end{proof}

\subsection{Filtrations: general definitions}

Let $\mathcal{M} = \langle W, R_1, \dots, R_n, \vartheta \rangle$ be a Kripke model and $\Gamma$ a set of formulas closed under subformulas. An equivalence relation $\sim$ is set to have a finite index if the quotient set $W / \sim$ is finite. The equivalence relation $\sim_{\Gamma}$ induced by $\Gamma$ is defined as

\begin{center}
  $w \sim_{\Gamma} v \Leftrightarrow \forall \varphi \in \Gamma \: (\mathcal{M}, w \models \varphi \Leftrightarrow \mathcal{M}, v \models \varphi)$.
\end{center}

If $\Gamma$ is finite, then $\sim_{\Gamma}$ has a finite index. An equivalence relation $\sim$ respects $\sim_{\Gamma}$, if $w \sim v$ implies $w \sim_{\Gamma} v$.

\begin{defin}
  Let $\mathcal{M} = \langle W, R_1, \dots, R_n, \vartheta \rangle$ be a Kripke model and $\Gamma$ be a Sub-closed set formulas. A $\Gamma$-filtration of $\mathcal{M}$ is a model
  $\widehat{\mathcal{M}} = \langle \widehat{W}, \widehat{R_1}, \dots, \widehat{R_n}, \widehat{\vartheta} \rangle$ such that:
  \begin{enumerate}
    \item $\widehat{W} = W / \sim$, where $\sim$ is an equivalence relation having a finite index that respects $\Gamma$
    \item $\widehat{\vartheta}(p) = \{ [x]_{\sim} \: | \: x \in W \: \& \: x \in \vartheta(p)\}$
    \item For each $i \in I$ one has $\widehat{R}_i^{\text{min}} \subseteq \widehat{R}_i \subseteq \widehat{R}_i^{\text{max}}$. $\widehat{R}_{i, \sim}^{\text{min}}$ is the $i$-th minimal filtered relation on $\widehat{W}$ defined as
    \begin{center}
      $\hat{x} \widehat{R}_{i, \sim}^{\text{min}} \hat{y} \Leftrightarrow \exists x' \sim x \: \exists y' \sim y \: x R_i y$
    \end{center}
    $\widehat{R}_{\Gamma,i}^{\text{max}}$ is the $i$-th maximal filtered relation on $\widehat{W}$ induced by $\Gamma$ defined as
    \begin{center}
      $\hat{x} \widehat{R}_{\Gamma,i}^{\text{max}} \hat{y} \Leftrightarrow \forall \Box_i \varphi \in \Gamma \: (\mathcal{M}, x \models \Box_i \varphi \Rightarrow \mathcal{M}, y \models \varphi)$
    \end{center}
  \end{enumerate}
\end{defin}

If $\Phi$ is finite subset of $\Gamma$ and $\sim = \sim_{\Phi}$, then $\widehat{\mathcal{M}}$ is a definable $\Gamma$-filtration of $\mathcal{M}$ through $\Phi$. If $\sim = \sim_{\Gamma}$, then such a filtration by means of the definiton above is called \emph{strict}.

\begin{lemma}
  Let $\Gamma$ be a finite set of formulas closed under subformulas and $\widehat{\mathcal{M}}$ a filtration of $\mathcal{M}$ through $\Gamma$, then for each $x \in W$ and for each $\varphi \in \Gamma$ one has
  \begin{center}
    $\mathcal{M}, x \models \varphi \Leftrightarrow \widehat{\mathcal{M}}, \hat{x} \models \varphi$
  \end{center}
\end{lemma}

\begin{defin} Let $\mathbb{F}$ be a class of Kripke frames and $\Gamma$ a finite set of formulas closed under subformulas. If for every model $\mathcal{M}$ over $\mathcal{F} \in \mathbb{F}$ there exists a model that is a $\Gamma$-definable filtration of $\mathcal{M}$, then $\mathbb{F}$ admits definable filtration. A class of models $\mathbb{M}$ admits definable filtration if for every $\mathcal{M} \in \mathbb{M}$ there exists a model belonging to the same class that is a definable $\Gamma$-filtration of $\mathcal{M}$.
\end{defin}

\begin{lemma}
  $ $

\begin{enumerate}
  \item Let $\mathcal{L}$ be a complete normal modal logic. If $\operatorname{Frames}(\mathcal{L})$ admits filtration, then $\mathcal{L}$ has the finite model property.
  \item If the class of models $\operatorname{Mod}(\mathcal{L})$ admits filtration, then $\mathcal{L}$ has the finite model property and Kripke complete as well.
\end{enumerate}
\end{lemma}

\section{Filtration of Euclidean logics}

First of all, let us ensure that a minimal filtration of an Euclidean frame is not necessary Euclidean.
Let $[x] \sim_{\Gamma} [y]$ and $[x] \sim_{\Gamma} [z]$. Then for some $x' \in [x]$ $y' \in [y]$, one has
$x' R y'$ and $x'' R z'$ for some $x'' \in [x]$ and $z' \in [z]$. Clearly, we cannot claim that $x' = x''$ in general. Thus, minimal filtration does not preserve the required property.

\begin{lemma}
  ${\bf K5}$ admit filtration.
\end{lemma}

\begin{proof}
  Let $\mathcal{M}$ be a ${\bf K5}$-model and $\Gamma_0$ a finite set of formulas closed under subformulas.
  Let us put $\Gamma = \Gamma_0 \cup \operatorname{Sub}(\{ \Diamond \Box \psi \: | \: \Box \psi \in \Gamma_0 \}) \cup \Psi$, where $\Psi = \nabla_1 \nabla_2 \dots \nabla_n \Box \psi$ for $\Box \psi \in \Gamma_0$ and
  $\nabla_i \in \{ \Diamond, \Box \}$. By Proposition~\ref{K5proves}, any element of $\Phi$ has one of the four forms. Thus, $W \sim_{\equiv_{\Gamma}}$ has a finite index.
  We put $\widehat{R} = R^{\operatorname{max}}_{\Gamma}$.
\end{proof}

\begin{defin} A first-order formula is called Horn if it has the following form:

  \begin{center}
    $\forall x_1, \dots, x_n (x_{i_1} R x_{j_1} \land \dots \land x_{i_s} R x_{j_s} \rightarrow x_k R x_l)$
  \end{center}
\end{defin}

\begin{defin}
  Let $H$ be a Horn property and $\langle W, R \rangle$ a Kripke frame. A Horn closure of a binary relation $R$ is the minimal relation $R^{H}$ containing $R$ and satisfying $H$.
\end{defin}

\begin{lemma}
  $R^{H} = \bigcup \limits_{n < \omega} R_n$ where

  \begin{enumerate}
    \item $R_0 = R$.
    \item $R_{n + 1} = R_n \cup \{ (a, b) \in W \: | \: \exists \vec{c} \in W \: P(a, b, \vec{c})\}$, where $P$ is a premise of $H$.
  \end{enumerate}
\end{lemma}

$E$-closure (an Euclidean Horn closure of a binary relation) has the following equivalent definitions:
\begin{lemma} \label{equivHorn}
  Let $\mathcal{F} = \langle W, R \rangle$ be a Kripke frame.
  The following conditions are equivalent:

  \begin{enumerate}
    \item $R^{E}$ is the smallest Euclidean relation containing $R$.
    \item $R^{E} = \bigcup \limits_{i < \omega} R_i$, where
    \begin{itemize}
      \item $R_0 = R$
      \item $R_{n + 1} = R_n \cup (R^{-1}_n \circ R_n)$
    \end{itemize}
    \item $x R^E y$ iff there exists $n < \omega$ such that
    either $x R y$ or $\exists z_1, \dots, z_n$ with $z_1 R x$ and $z_{n - 1} R y$ and for each $1 < i \leq n$ one has
    either $z_{i - 1} R z_i$ or $z_i R z_{i - 1}$.
    \item $R^{E} = R \cup \bigcup \limits_{i < \omega} (R^{-1} \circ (R \circ R^{-1})^n \circ R)$.
  \end{enumerate}
\end{lemma}

\begin{proof}
$ $

\begin{enumerate}
  \item $(1) \Rightarrow (2)$
  Let us show that if $R^{E}$ is the smallest Euclidean relation containing $R$, then $R^{E} = \bigcup \limits_{i < \omega} R_i$. There are two inclusions:
  \begin{itemize}
    \item $R^{E} \subseteq \bigcup \limits_{i < \omega} R_i$. Recall that $R^{E}$ has the form (?):
    \begin{center}
      $R^{E} = \bigcap \{ R' \: | \: R \subseteq R', \forall a, b \in W \: R'(a,b) \Rightarrow \exists x \in W \: R'(x, a) \: \& \: R'(x, b)\}$
    \end{center}

    \item $\bigcup \limits_{i < \omega} R_i \subseteq R^{E}$.
    Let us show that $x R_n y$ for each $n < \omega$ implies $x R^E y$ by induction on $n$.
    If $n = 0$, then $x R y$, thus, $x R^{E} y$, since $R$ is a subrelation of $R^{E}$.
    Suppose $n = m + 1$ and $x R_{m + 1} y$. Let us show that $x R^{E} y$.
    From $x R_{m + 1} y$, one has $(x, y) \in R^n \cup (R^{-1}_n \circ R_n)$. There are two cases:
    \begin{itemize}
      \item $x R^n y$, one needs to merely apply the IH.
      \item $x R^{-1}_n \circ R_n y$. Then $\exists z \in W \: x R^{-1}_n z \: \& \: z R_n$. That is,
      $z R_n x$ and $z R_n y$ for some $z$. $R_n$ is already a subrelation of $R^E$. Thus,
      $z R^{E} x$ and $z R^{E} y$. That implies $x R^{E} y$.
    \end{itemize}
  \end{itemize}
  \item $(2) \Rightarrow (3)$
  Let $(x, y) \in R_m$, let us the statement by induction on $m$.
  \begin{enumerate}
    \item Suppose $m = 0$, then $x R y$, and the statement is shown putting $n = 0$.
    \item Suppose $m = p + 1$ and $x R_{p + 1} y$.
    Assume that either $x R y$ or $\exists z_1, \dots, z_p$ with $z_1 R x$ and $z_{p - 1} R y$ and for each $1 < i \leq p$ one has
    either $z_{i - 1} R z_i$ or $z_i R z_{i - 1}$.

    $x R_{p + 1} y$ implies $(x, y) \in R_{p} \cup (R^{-1}_p \circ R_p)$. If $(x, y) \in R_{p}$, then we merely apply the IH.
    Suppose $(x, y) \in R^{-1}_p \circ R_p$, then $(z, x) \in R_p$ and $(z, y) \in R_p$
  \end{enumerate}
  \item $(3) \Rightarrow (4)$
  Suppose either $x R y$ or there exist $n \geq 1$ and $z_1, \dots, z_n$ with $z_1 R x$ and $z_{n - 1} R y$ and for each $1 < i \leq n$ one has either $z_{i - 1} R z_i$ or $z_i R z_{i - 1}$.
  If $x R y$, then we are done. Otherwise there exists $n \geq 1$ with the condition above. Then $(x, y) \in R_{n + 1}$ that follows from the condition.
  \item $(4) \Rightarrow (1)$
\end{enumerate}
\end{proof}

\begin{lemma}
 Let $\mathcal{F} = \langle W, R \rangle$ be a Kripke frame. Let us define $R^{E} = \bigcup \limits_{i < \omega} R_i$ where:

 \begin{enumerate}
   \item $R_0 = R$
   \item $R_{n + 1} = R_n \cup ({R_n}^{-1} \circ R_n)$
 \end{enumerate}
 Then $R^{E}$ is Euclidean.
\end{lemma}

\begin{proof}
  Let $(x, y), (x, z) \in R^{E}$, one needs to show that $(y, z) \in R^{E}$.
  Clearly that $(x, y) \in R_i$ and $(x, y) \in R_j$ for some $i, j < \omega$. Thus, we need $(y, z) \in R_m$ for some $m$ depending on $i$ and $j$.

  Let us consider the following cases:
  \begin{enumerate}
    \item $i = 0$ and $j = 0$

    Suppose $(x, y), (x, z) \in R_0 = R$, then $(y, z) \in R^{-1} \circ R$. Thus, $(y, z) \in R_1$
    \item $i = 0$ and $j = k + 1$

    Suppose $(x, y) \in R$ and $(x, z) \in R_{k + 1} = R_{k} \cup ({R_k}^{-1} \circ R_k)$.
    Clearly that $(x, y) \in R_{k + 1}$ as well. It is obviously that $(y, z) \in R_{k + 2}$ since
    $(y, x) \in R_{k + 1}^{-1}$ and $(x, z) \in R_{k + 1}$.

    \item The case with $i = k + 1$ and $j = 0$ is similar to the previous one.

    \item Suppose $i = m + 1$ and $i = k + 1$. That is, $(x, y) \in R_{m + 1} = R_m \cup ({R_m}^{-1} \circ R_m)$ and $(x, z) \in R_{k + 1} = R_k \cup ({R_k}^{-1} \circ R_k)$.
    Consider the following four subcases:
    \begin{enumerate}
      \item Suppose $(x, y) \in R_m$ and $(x, z) \in R_k$ and $m \leq k$ without loss of generality.
      $m \leq k$ implies $R_m \subseteq R_k$ and $(x, y) \in R_k$ in particular. Thus, $(y, z) \in {R_k}^{-1} \circ R_k$, so $(y, z) \in R_{k + 1}$.
      \item The rest of the cases are similar to the first one.
    \end{enumerate}
    \end{enumerate}
\end{proof}

\begin{lemma}
  Let $\mathcal{M} = \langle W, R, \vartheta \rangle$ be an Euclidean model, $\Gamma$ a set of $\operatorname{Sub}$-closed formulas, and $\sim$ an equivalence relation having a finite index that respects $\Gamma$, then $\widehat{R} = (R^{min}_{\Phi})^{E} \subseteq R^{max}_{\Gamma}$, where $\Phi = \Gamma \cup \{ \Diamond \Box \varphi \: | \: \Box \varphi \in \Gamma \}$.

  Thus, ${\bf K}5$ admits strict filtrations.
\end{lemma}

\begin{proof}
  Recall that $(R^{min}_{\Phi})^{E}$ has the form $(R^{min}_{\Phi})^{E} = \bigcup \limits_{n < \omega} (R^{min}_{\Phi})_n$, where
  \begin{enumerate}
    \item $(R^{min}_{\Phi})_0 = R^{min}_{\Phi}$
    \item $(R^{min}_{\Phi})_{m + 1} = (R^{min}_{\Phi})_n \cup (((R^{min}_{\Phi})_n)^{-1} \circ (R^{min}_{\Phi})_n)$
  \end{enumerate}
  One needs to show that for each $n < \omega$ $(R^{min}_{\Phi})_n \subseteq R^{max}_{\Gamma}$. We prove this by induction.
  Suppose $\mathcal{M}, x \models \Box \varphi$ for $\Box \varphi \in \Phi$ and $[x] (R^{min}_{\Phi})^{E} [y]$. We need $\mathcal{M}, y \models \varphi$.
  \begin{enumerate}
    \item $([x], [y]) \in (R^{min}_{\Phi})_0$, then $([x], [y]) \in R^{min}_{\Phi}$. Then there exist $x' \in [x]$ and $y' \in [y]$ such that $x' R y'$. So $\mathcal{M}, x' \models \Box \varphi$ and, thus,
    $\mathcal{M}, y' \models \varphi$. Then $\mathcal{M}, y' \models \varphi$ as well since $y' \in [y]$.
    \item $([x], [y]) \in (R^{min}_{\Phi})_{m + 1}$, then
    $([x], [y]) \in (R^{min}_{\Phi})_{m} \cup (((R^{min}_{\Phi})_{m})^{-1} \circ R^{min}_{\Phi})_{m})$.

    If $([x], [y]) \in (R^{min}_{\Phi})_{m}$, then we apply the IH.

    Suppose $([x], [y]) \in (R^{min}_{\Phi})_{m})^{-1} \circ (R^{min}_{\Phi})_{m}$,
    then there exists $[z] \in W / \sim_{\Phi}$ such that $([z], [x]) \in (R^{min}_{\Phi})_{m})$ and $([z], [y]) \in (R^{min}_{\Phi})_{m}$.

    Then one has the following picture (using Lemma~\ref{equivHorn}):

    \xymatrix{
&&    [z] & \ar[l]_{R^{min}_{\Phi}} [z_1] \ar@{-}[r]^{R'} & [z_2] \ar@{-}[r]^{R'} & \ar@{-}[r]^{R'} \dots & [z_{m - 1}] \ar@{-}[r]^{R'} & [z_m] \ar[r]^{R^{min}_{\Phi}} & [x] \\
&&    [z] & \ar[l]^{R^{min}_{\Phi}} [z_1^{'}] \ar@{-}[r]_{R'} & [z_2^{'}] \ar@{-}[r]_{R'} & \ar@{-}[r]_{R'} \dots & [z_{m - 1}] \ar@{-}[r]_{R'} & [z_m^{'}] \ar[r]_{R^{min}_{\Phi}} & [y]
    }
  Where $R'$ is either $R^{min}_{\Phi}$ or its converse.
  One has $\mathcal{M}, x \models \Box \varphi$ for $\Box \varphi \in \Phi$, where $\widehat{M}$ is the minimal filtration of $\mathcal{M}$ through $\Phi$. One has $[z_m] R^{min}_{\Phi} [x]$, then $a_m R a$ for some $a_m \in [z_n]$ and $a \in [x]$. Thus, $\mathcal{M}, a_m \models \Diamond \Box \varphi$ and, thus, $\widehat{\mathcal{M}}, [z_m] \models \Diamond \Box \varphi$.

  Applying Proposition~\ref{k5useful} several times, one may show that $\widehat{\mathcal{M}}, [z_1] \models \Diamond \Box \varphi$.
  One has $[z_1] R^{min}_{\Phi} [z]$, then for some $a \in [z]$ and $a_1 \in [z_1]$ we have $a_1 R a$.

  Then $\mathcal{M}, a \models \Box \varphi$ and $\widehat{M}, [z] \models \Box \varphi$.

  We have $[z_1^{'}] R^{min}_{\Phi} [z]$, thus, $a_1^{'} R a^{'}$ for some $a_1^{'} \in [z_1^{'}]$ and $a^{'} \in [z]$. Then
  $\mathcal{M}, a_1^{'} \models \Diamond \Box \varphi$, and, thus, $\widehat{M}, [z_1^{'}] \models \Diamond \Box \varphi$.

  One may show that $\widehat{M}, [z_{m}^{'}] \models \Diamond \Box \varphi$ in the same way via Lemma~\ref{k5useful}. Thus, $\mathcal{M}, z_{m}^{'} \models \Diamond \Box \varphi$. We also have $\mathcal{M}, z_{m}^{'} \models \Diamond \Box \varphi \to \Box \varphi$, and, thus,
  $\mathcal{M}, z_{m}^{'} \models \Box \varphi$. Then $\widehat{\mathcal{M}}, [z_m^{'}] \models \Box \varphi$.

  One has $[z_m^{'}] R^{min}_{\Phi} [y]$, then $a_m^{'} R y'$ for some $a_m^{'} \in [z_m^{'}]$ and $y' \in [y]$. Then $\mathcal{M}, y' \models \varphi$. But $y' \sim_{\Phi} y$, so $\mathcal{M}, y \models \varphi$.
  \end{enumerate}
\end{proof}


\section{Filtration for {\bf K4}}

\begin{prop}
  Let $R$ be a binary relation on $W \neq \emptyset$. Define $R^{+} = \bigcup \limits_{i < \omega} R_i$
  \begin{enumerate}
    \item $R_0 = R$
    \item $R_{n + 1} = R_n \circ R$
  \end{enumerate}
  Then $R^{+}$ is transitive
\end{prop}

\begin{lemma}
  Let $\mathcal{M} = \langle W, R, \vartheta \rangle$ be a transitive model and $\overline{\mathcal{M}} = \langle \overline{W}, \overline{R}, \overline{\vartheta} \rangle$ its minimal filtration through a finite $\operatorname{Sub}$-closed set of formulas $\Theta$.

  Then $\overline{\mathcal{M}}^{+} = \langle \overline{W}, (\overline{R})^{+}, \overline{\vartheta} \rangle$ is a $\Theta$-filtration of $\mathcal{M}$.
\end{lemma}

\begin{proof}
  $(\overline{R})^{+}$ obviously contains $R$. By the previous proposition, $(\overline{R})^{+}$ is transitive.
  Let us show that $(\overline{R})^{+} \subseteq R^{max}_{\Theta}$.

  Let $\hat{x}, \hat{y} \in \overline{W}$ with $\hat{x} (\overline{R})^{+} \hat{y}$ and
  $\Box \varphi \in \Theta$ with $\mathcal{M}, x \models \Box \varphi$. Let us show that $\mathcal{M}, y \models \varphi$.

  If $\hat{x} (\overline{R})^{+} \hat{y}$, then there exist equivalence classes $\hat{x}_1, \dots, \hat{x}_n$ such that
  \begin{center}
    $\hat{x} \overline{R} \hat{x}_1 \overline{R} \dots \overline{R} \hat{x}_n \overline{R} \hat{y}$
  \end{center}

  $\mathcal{M}, x \models \Box \varphi$ implies $\mathcal{M}, x \models \Box \varphi \to \Box \Box \varphi$. Thus, $\overline{M}, \hat{x} \models \Box \Box \varphi$.

  $\hat{x} \overline{R} \hat{x}_1$, so there are $x_1 \in \hat{x}$ and $x_2 \in \hat{x}_1$ with $x_1 R x_2$. In particular, $\mathcal{M}, x_2 \models \Box \varphi$,
  so $\overline{\mathcal{M}}, \hat{x}_2 \models \Box \varphi$, and et cetera.

  For each $i \in \{1, \dots, n\}$ we have $\mathcal{M}, x_i \models \Box \varphi$ which is shown inductively:

  If $\mathcal{M}, x_i \models \Box \varphi$ for $x_i \in \hat{x}_i$, so $\mathcal{M}, x_i \models \Box \Box \varphi$, but there exist ${x_i}' \in \hat{x}_i$ and $x_{i + 1} \in \hat{x}_{i + 1}$, so
  $\mathcal{M}, x_{i + 1} \models \Box \varphi$.

  Finally, we have $\mathcal{M}, x_n \models \Box \varphi$ for $x_n \in \hat{x}_n$, but $\hat{x}_n \overline{R} \hat{y}$, so $\mathcal{M}, y' \models \varphi$ for each $y' \in \hat{y}$. Thus, $\varphi$ is true at $y$ as well.
\end{proof}

\bibliographystyle{plain}
\bibliography{Text}

\end{document}

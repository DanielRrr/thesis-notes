\documentclass[a4paper]{article}
\usepackage{amsmath}
\usepackage{amsthm}
\usepackage{amsfonts}
\usepackage{amssymb}
\usepackage{bussproofs}
\usepackage{mathtools}
\usepackage{verbatim}
\usepackage{dsfont}
\usepackage{mathabx}
\usepackage[all, 2cell]{xy}
\usepackage[all]{xy}
\usepackage{wasysym}
\usepackage{rotating}
\usepackage{geometry}
\usepackage{trfsigns}
\usepackage{cmll}
\usepackage{authblk}
\usepackage{hyperref}
\usepackage{cleveref}
\usepackage{lipsum}
\usepackage{extpfeil}
\usepackage{soul}
\usepackage{graphicx}

\newcommand\mapsfrom{\mathrel{\reflectbox{\ensuremath{\mapsto}}}}

\theoremstyle{defin}
\newtheorem{defin}{Definition}

\theoremstyle{theorem}
\newtheorem{theorem}{Theorem}

\theoremstyle{prop}
\newtheorem{prop}{Proposition}

\theoremstyle{lemma}
\newtheorem{lemma}{Lemma}

\theoremstyle{ex}
\newtheorem{ex}{Example}


\theoremstyle{col}
\newtheorem{col}{Corollary}

\author{Daniel Rogozin}
\date{}
\title{Note on filtration of logics containing {\bf K5}}

\begin{document}

\maketitle

\nocite{*}

\section{Preliminaries}

\begin{defin}
  An $n$-normal modal logic is a set of formulas that contains all Boolean tautologies, formulas $\Diamond_i p \lor \Diamond_i q \leftrightarrow \Diamond_i (p \lor q)$ and $\Diamond_i \bot \leftrightarrow \bot$ for $i \leq n$, and is closed under modus ponens, substitution, and monotonicity: from $\varphi \rightarrow \psi$ infer $\Diamond_i \varphi \rightarrow \Diamond_i \psi$ for $i \leq n$.
\end{defin}

\begin{defin} An $n$-Kripke model is a triple $\mathcal{M} = \langle W, R_1, \dots, R_n, \vartheta \rangle$, where $R_i \subseteq W \times W$, $\vartheta : \operatorname{PV} \to 2^W$, and the connectives have the following semantics:

  \begin{enumerate}
    \item $\mathcal{M}, w \models p \Leftrightarrow w \in \vartheta(p)$
    \item $\mathcal{M}, w \models \not \varphi \Leftrightarrow \mathcal{M}, w \nvDash \varphi$
    \item $\mathcal{M}, w \models \varphi \lor \psi \Leftrightarrow \mathcal{M}, w \models \varphi \text{ or } \mathcal{M}, w \models \psi$
    \item $\mathcal{M}, w \models \Diamond_i \varphi \Leftrightarrow \exists v \in R_i(w) \: \mathcal{M}, v \models \varphi$
  \end{enumerate}
\end{defin}

By ${\bf K5}$ we mean the logic ${\bf K} \oplus {A5}$, where ${A5} = \Diamond p \to \Box \Diamond p$. It is known that ${\bf K5}$ is the modal logic of all Euclidean frames. A frame is called Euclidean if for each $x, y, z$, $x R y$ and $x R z$ implies $y R z$.

\begin{prop}
  Let $\mathcal{F} = \langle W, R \rangle$ be an Euclidean frame.

  \begin{enumerate}
    \item For each $x, y, z \in W$, $x R y$ and $x R z$ implies either $y R z$ or $z R y$.
    \item $R \subseteq R ; R$, that is, $R$ is dense.
    \item For each $x \in W$, $R^{*}(x) = \{ x\} \cup R(R(x))$.
  \end{enumerate}
\end{prop}

Let $\mathcal{M} = \langle W, R_1, \dots, R_n, \vartheta \rangle$ be a Kripke model and $\Gamma$ a set of formulas closed under subformulas. An equivalence relation $\sim$ is set to have a finite index if the quotient set $W / \sim$ is finite. The equivalence relation $\sim_{\Gamma}$ induced by $\Gamma$ is defined as

\begin{center}
  $w \sim_{\Gamma} v \Leftrightarrow \forall \varphi \in \Gamma \: (\mathcal{M}, w \models \varphi \Leftrightarrow \mathcal{M}, v \models \varphi)$.
\end{center}

If $\Gamma$ is finite, then $\sim_{\Gamma}$ has a finite index. An equivalence relation $\sim$ respects $\sim_{\Gamma}$, if $w \sim v$ implies $w \sim_{\Gamma} v$.

\begin{defin}
  Let $\mathcal{M} = \langle W, R_1, \dots, R_n, \vartheta \rangle$ be a Kripke model and $\Gamma$ be a Sub-closed set formulas. A $\Gamma$-filtration of $\mathcal{M}$ is a model
  $\widehat{\mathcal{M}} = \langle \widehat{W}, \widehat{R_1}, \dots, \widehat{R_n}, \widehat{\vartheta} \rangle$ such that:
  \begin{enumerate}
    \item $\widehat{W} = W / \sim$, where $\sim$ is an equivalence relation having a finite index that respects $\Gamma$
    \item $\widehat{\vartheta}(p) = \{ [x]_{\sim} \: | \: x \in W \: \& \: x \in \vartheta(p)\}$
    \item For each $i \in I$ one has $\widehat{R}_i^{\text{min}} \subseteq \widehat{R}_i \subseteq \widehat{R}_i^{\text{max}}$. $\widehat{R}_{i, \sim}^{\text{min}}$ is the $i$-th minimal filtered relation on $\widehat{W}$ defined as
    \begin{center}
      $\hat{x} \widehat{R}_{i, \sim}^{\text{min}} \hat{y} \Leftrightarrow \exists x' \sim x \: \exists y' \sim y \: x R_i y$
    \end{center}
    $\widehat{R}_{\Gamma,i}^{\text{max}}$ is the $i$-th maximal filtered relation on $\widehat{W}$ induced by $\Gamma$ defined as
    \begin{center}
      $\hat{x} \widehat{R}_{\Gamma,i}^{\text{max}} \hat{y} \Leftrightarrow \forall \Box_i \varphi \in \Gamma \: (\mathcal{M}, x \models \Box_i \varphi \Rightarrow \mathcal{M}, y \models \varphi)$
    \end{center}
  \end{enumerate}
\end{defin}

If $\Phi$ is finite subset of $\Gamma$ and $\sim = \sim_{\Phi}$, then $\widehat{M}$ is a definable $\Gamma$-filtration of $\mathcal{M}$ through $\Phi$. If $\sim = \sim_{\Gamma}$, then such a filtration by means of the definiton above is called \emph{strict}.

\begin{lemma}
  Let $\Gamma$ be a finite set of formulas closed under subformulas and $\widehat{M}$ a filtration of $\mathcal{M}$ through $\Gamma$, then for each $x \in W$ and for each $\varphi \in \Gamma$ one has
  \begin{center}
    $\mathcal{M}, x \models \varphi \Leftrightarrow \widehat{M}, \hat{x} \models \varphi$
  \end{center}
\end{lemma}

\begin{defin} Let $\mathbb{F}$ be a class of Kripke frames and $\Gamma$ a finite set of formulas closed under subformulas. If for every model $\mathcal{M}$ over $\mathcal{F} \in \mathbb{F}$ there exists a model that is a $\Gamma$-definable filtration of $\mathcal{M}$, then $\mathbb{F}$ admits definable filtration. A class of models $\mathbb{M}$ admits definable filtration if for every $\mathcal{M} \in \mathbb{M}$ there exists a model belonging to the same class that is a definable $\Gamma$-filtration of $\mathcal{M}$.
\end{defin}

\begin{lemma}
  $ $

\begin{enumerate}
  \item Let $\mathcal{L}$ be a complete normal modal logic. If $\operatorname{Frames}(\mathcal{L})$ admits filtration, then $\mathcal{L}$ has the finite model property.
  \item If the class of models $\operatorname{Mod}(\mathcal{L})$ admits filtration, then $\mathcal{L}$ has the finite model property and Kripke complete as well.
\end{enumerate}
\end{lemma}

\section{Filtration of Euclidean logics}

First of all, let us ensure that a filtration of an Euclidean frame is not necessary finite.
Let $[x] \sim_{\Gamma} [y]$ and $[x] \sim_{\Gamma} [z]$. Then for some $x' \in [x]$ $y' \in [y]$, one has
$x' R y'$ and $x'' R z'$ for some $x'' \in [x]$ and $z' \in [z]$. Clearly, we cannot claim that $x' = x''$ in general. Thus, minimal filtration does not preserve the required property.

\subsection{Clusters for transitive frames}

\subsection{Clusters for transitive Euclidean frames}

\subsection{Clusters for Euclidean frames}

\section{Transitive closure stuff}

\section{PDLisation of Euclidean logics}

\section{Transitive closure and fusion}

\bibliographystyle{plain}
\bibliography{Text}

\end{document}

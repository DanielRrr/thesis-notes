\documentclass[a4paper]{article}
\usepackage{amsmath}
\usepackage{amsthm}
\usepackage{amsfonts}
\usepackage{amssymb}
\usepackage{bussproofs}
\usepackage{mathtools}
\usepackage{verbatim}
\usepackage{dsfont}
\usepackage{mathabx}
\usepackage[all, 2cell]{xy}
\usepackage[all]{xy}
\usepackage{wasysym}
\usepackage{rotating}
\usepackage{geometry}
\usepackage{trfsigns}
\usepackage{cmll}
\usepackage{authblk}
\usepackage{hyperref}
\usepackage{cleveref}
\usepackage{lipsum}
\usepackage{extpfeil}
\usepackage{soul}
\usepackage{graphicx}

\newcommand\mapsfrom{\mathrel{\reflectbox{\ensuremath{\mapsto}}}}

\theoremstyle{defin}
\newtheorem{defin}{Definition}

\theoremstyle{theorem}
\newtheorem{theorem}{Theorem}

\theoremstyle{claim}
\newtheorem{claim}{Claim}

\theoremstyle{prop}
\newtheorem{prop}{Proposition}

\theoremstyle{lemma}
\newtheorem{lemma}{Lemma}

\theoremstyle{fact}
\newtheorem{fact}{Fact}

\theoremstyle{ex}
\newtheorem{ex}{Example}


\theoremstyle{col}
\newtheorem{col}{Corollary}

\let\strokeL\L
\DeclareRobustCommand{\L}{\ifmmode\mathbf{L}\else\strokeL\fi}

\author{Daniel Rogozin}
\date{}
\title{Complete representability for canonical extensions of representable cylindric algebras}

\begin{document}
\maketitle

\section{The problem itself}

\begin{itemize}
\item Given $\mathcal{C} \in {\bf RCA}_{\omega}$, whether $\mathcal{C}^{+}$ has a complete, $\omega$-dimensional representation? The conjecture is yes. \cite{hirsch2002relation}
\item Whether ${\bf RCA}_{\omega}$ is barely canonical. The conjecture is yes.
\end{itemize}
\section{Atomic Representations}

A representation of a Boolean algebra $\mathcal{B}$ is an embedding $h$ of $\mathcal{B}$ to some field of sets.

Let $a \in \mathcal{B}$ be an element of a Boolean algebra $\mathcal{B}$, $a$ is called an atom, if for every $b \in \mathcal{B}$
$b < a$ implies $b = 0$. That is, an atom is a minimal non-zero element. $\operatorname{At}(\mathcal{B})$ is the set of all atoms
of $\mathcal{B}$.

Let $\mathcal{B}$ be a Boolean algebra and $\mathcal{F}$ a field of sets such that $h : \mathcal{B} \to \mathcal{F}$ is a
representation of $\mathcal{B}$, then $\mathcal{B}$ is a complete representation of $\mathcal{B}$, if for every
$A \subseteq \mathcal{B}$ we have the following whenever $\Sigma A$ is defined:
\begin{center}
  $h(\Sigma A) = \bigcup h[A]$
\end{center}

A representation $h$ is called atomic, if $x \in h(1)$ there exists $b \in \operatorname{At}(\mathcal{B})$ such that $x \in h(b)$.

\begin{theorem} \label{completeboolean}
  Let $\mathcal{B}$ be a Boolean algebra, then $\mathcal{B}$ is atomic iff $\mathcal{B}$ is completely representable. See \cite[Corollary 6]{hirsch1997complete}.
\end{theorem}

\section{BAOs and Duality}

By default, we assume that all operators are at most unary. Here is the rigorous definition:

\begin{defin}

$ $

  \begin{enumerate}
    \item Let $\mathcal{B} = \langle B, +, -, 0, 1 \rangle$ be a Boolean algebra. An operator is a function $\Omega : B \to B$ satisfying the following conditions:
    \begin{itemize}
      \item Normality: $\Omega(0) = 0$
      \item Additivity: $\Omega(b + b') = \Omega(b) + \Omega(b')$
    \end{itemize}
    \item Let $I$ be an index set, a Boolean algebra with operators (BAO) is an algebra $\langle B, +, -, 0, 1, (\Omega_{i})_{i \in I} \rangle$ such that $\langle B, +, -, 0, 1 \rangle$ is a Boolean algebra and for each $i \in I$ $\Omega_{i}$ is an operator.
  \end{enumerate}
\end{defin}

\begin{defin} Let $\mathcal{B} = \langle B, +, -, 0, 1, (\Omega_{i})_{i \in I} \rangle$ be a BAO, then

  \begin{enumerate}
    \item An opetator $\Omega$ is completely additive, if for every $X \subseteq B$ such that $\Sigma X$ is defined, one has

    \begin{center}
      $\Omega(\sum X) = \sum \limits_{x \in X} \Omega(x)$
    \end{center}
    \item $\mathcal{B}$ is completely additive, if for each $i \in I$ $\Omega_{i}$ is additive,
    \item A class $\mathcal{K}$ of BAOs is completely additive, if every $\mathcal{B} \in \mathcal{K}$ is completely additive.
  \end{enumerate}
\end{defin}

\subsection{Atom structures and canonical extensions}

\begin{defin}  Let $I$ be an index set and $(\Omega_i)_{i \in I}$ a set of function symbols
\begin{enumerate}
  \item A structure is a relational structrure
  $\mathcal{F} = \langle W, (R_{i})_{i \in I} \rangle$
  such that $R_{i}$ is a binary relation symbol for a function symbol $\Omega_{i \in I}$ with the corresponding index,
  \item Let $\mathcal{B}$ be an atomic BAO of the signature $I$,
the atom structure of $\mathcal{B}$, written as $\mathfrak{At} \mathcal{B}$, is a structure $\langle \operatorname{At}(\mathcal{B}), (R_{i})_{i \in I} \rangle$ such that for all
$a, b \in \operatorname{At}(\mathcal{B})$ and for all $i \in I$
\begin{center}
  $\mathfrak{At} \mathcal{B} \models R_{i}(a,b)$ iff $\mathcal{B} \models a \leq \Omega_{i}(b)$
\end{center}
\item Let $\mathcal{F} = \langle W, (R_{i})_{i \in I} \rangle$ be an atom structure, the complex algebra of $\mathcal{F}$, written as $\mathfrak{Cm} \mathcal{F}$, is a BAO
$\langle \mathcal{P}(W), \cup, -, \emptyset, W, (\Omega_{R_{i}})_{i \in I} \rangle$ such that
for all $X \subseteq W$ and for each $i \in I$:
\begin{center}
  $\Omega_{R_{i}}(X) = \{ a \in W \: | \: \exists b \in X \: \mathcal{F} \models R_{i}(a, b)\}$
\end{center}
\end{enumerate}
\end{defin}

\begin{defin} Let $\mathcal{F} = \langle W, (R_{i})_{i \in I} \rangle$ and $\mathcal{F}' = \langle W', ({R'}_{i})_{i \in I} \rangle$, then a function $f : \mathcal{F} \to \mathcal{F}'$ is a bounded morphism, if the following holds:
\begin{enumerate}
\item $x R_i y$ implies $f(x) {R'}_i f(y)$;
\item $f(x) {R'}_i z$, then there exists $y \in W$ such that $x R_i y$ and $f(y) = z$.
\end{enumerate}
A bounded morphism $f : \mathcal{F} \to \mathcal{F}'$ is a $p$-morphism, if $f$ is onto. $\mathcal{F} \twoheadrightarrow \mathcal{F}'$ iff there exists a $p$-morphism from $\mathcal{F}$ onto $\mathcal{F}'$, or $\mathcal{F}'$ is a $p$-morphic image of $\mathcal{F}$.
\end{defin}

\begin{defin} Let $\mathcal{F} = \langle W, (R_{i})_{i \in I} \rangle$ is an inner substructure \footnote{Or alternatively, a generated subframe} of $\mathcal{F}' = \langle W', ({R'}_{i})_{i \in I} \rangle$, if $W \subseteq W'$ and the embedding $\mathcal{F} \hookrightarrow \mathcal{F}'$ is a bounded morphism. Let $\mathbb{F}$ be a class atom structures, then $\mathbb{S}(\mathbb{F})$ is the closure of $\mathbb{F}$ under generated subframes.
\end{defin}

Let $\mathbb{F}$ be a class of structures, define:
\begin{enumerate}
\item $\mathfrak{Cm}(\mathbb{F}) = \{ \mathcal{B} \: | \: \mathcal{B} \cong \mathfrak{Cm}(\mathcal{F}) \text{ for some $\mathcal{F} \in {\bf F}$}\}$.
\item ${\bf Up}(\mathbb{F})$ is the class of structures isomorphic to disjoint unions of elements of $\mathbb{F}$.
\item ${\bf S}(\mathbb{F})$ is the closure of $\mathbb{F}$ under inner substructures.
\end{enumerate}

Let $A$ be a non-empty subset of a Boolean algebra $\mathcal{B}$, $A$ is a \emph{filter}, if $A$ is closed under finite infima and it is upward closed. $A$ is an ultrafilter, if it has no non-trivial extensions. That is, if $A \subseteq A'$, then $A' = \mathcal{B}$. This is a well-known fact that every filter can be extended to a maximal one using Zorn's lemma.

The following definition is due to, for example, \cite[Definition 5.40]{venema2010}.

\begin{defin}
  Let $\mathcal{B} = \langle B, +, -, 0, 1, (\Omega_i)_{i \in I} \rangle$ be a BAO and ${\bf Spec}(\mathcal{B})$ the set of its ultrafilters. The ultrafilter frame of $\mathcal{B}$ (or the canonical frame) is a relational structure $\mathcal{F}_{\mathcal{B}} = \langle {\bf Spec}(\mathcal{B}), R_{\Omega_i} \rangle$ such that for all ultrafilters $U_1, U_2$ one has
  \begin{center}
    ${\bf Spec}(\mathcal{B}) \models R_{\Omega_i}(U_1, U_2)$ iff $\{ \Omega_i(b) \: | \: b \in U_1 \} \subseteq U_2$.
  \end{center}
\end{defin}

Given $\mathcal{B}$ be a BAO, we denoted as $\mathcal{B}^{+}$ as the complex algebra of the canonical frame $\mathfrak{Cm}(\mathcal{F}_{\mathcal{B}})$, that is, \emph{the canonical extension} of $\mathcal{B}$.
A class of BAOs ${\bf K}$ is canonical, if it is closed under canonical extensions. That is, $\mathcal{B}^{+} \in {\bf K}$ whenever $\mathcal{B} \in {\bf K}$.

\begin{theorem} Let $\mathcal{A}$, $\mathcal{B}$ be BAOs,

\begin{enumerate}
  \item There exists $\iota : \mathcal{A} \hookrightarrow \mathcal{A}^{+}$ such that
  $\iota : a \mapsto \{ \gamma \in {\bf Spec}(\mathcal{A}) \: | \: a \in \gamma \}$.
  \item $i : \mathcal{A} \hookrightarrow \mathcal{B}$ implies
  $i^{+} : \mathcal{A}^{+} \hookrightarrow \mathcal{B}^{+}$
\end{enumerate}
\end{theorem}

\section{Representable cylindric algebras}

Let $\alpha$ be an ordinal. Denote $\{ f \: | \: f \alpha \to U\}$ as $\prescript{\alpha}{}U$. $x_i$ stands for $x(i)$, where
$x \in \prescript{\alpha}{}U$ and $i < \alpha$.

A subset of $\prescript{\alpha}{}U$ is an $\alpha$-ry relation on $U$. For $i, j < \alpha$, the \emph{$i,j$-diagonal} $D_{ij}$ is the set of all elements of $\prescript{\alpha}{}U$ such that $y_i = y_j$.

If $i < \alpha$ and $X$ is an $\alpha$-ry relation on $U$, then
the $i$-th cylindrification $C_i X$ is the set of all elements of $U$ that agree with some element of $X$ on each coordinate except, perhaps, the $i$-th one. To be more precise,
\begin{center}
$C_i X = \{ y \in \prescript{\alpha}{}U \: |
\: \exists x \in X \forall i < \alpha \: (i \neq j \Rightarrow y_j = x_j)\}$.
\end{center}
We define the following equivalence relation for $i < \alpha$ and $x, y \in \prescript{\alpha}{}U$:
\begin{center}
 $x \equiv_i y \Leftrightarrow \forall j \in \alpha \: (i \neq j \Rightarrow x(i) = y(j))$
\end{center}
Then one may reformulate the definition of the $i$-th cylindrification in the following way:

\begin{center}
 $C_i X = \{ y \in \prescript{\alpha}{}U \: | \: \exists x \in X \:\: x \equiv_i y \}$
\end{center}

According to this version of the definiton, one may think of the cylindrification as an ${\bf S}5$ modal operator.

\begin{defin}
 A cylindic set algebra of dimension $\alpha$ is an algebra consisting of a set $S$ of $\alpha$-ry relation on some base set $U$
   with the constants and operations $0 = \emptyset$, $1 = \prescript{\alpha}{}U$, $\cap$, $-$, the diagonal elements $(D_{ij})_{i, j < \alpha}$, the cylindrifications $(C_i)_{i < \alpha}$. A generalised cylindric set algebra of dimension $\alpha$ is a subdirect of cylindric algebras that have dimension $\alpha$.
   ${\bf Cs}_{\alpha}$ denotes the class of all cylindric set algebras of dimension $\alpha$.
\end{defin}

\begin{defin}
   A cylindric algebra of dimension $\alpha$ is an algebra $\mathcal{C} = \langle \mathcal{B}, \{ c_i \}_{i < \alpha}, \{ d_{ij} \}_{i, j < \alpha} \rangle$ such that
   \begin{itemize}
     \item $\mathcal{B}$ is a Boolean algebra, for each $i, j < \alpha$ $c_i$ is an operator and $d_{ij} \in \mathcal{B}$
     \item For each $i < \alpha$, $a \leq c_i a$, $c_i (a \cdot c_i b) = c_i a \cdot c_i b$ and $d_{ii} = 1$
     \item For every $i, j < \alpha$, $c_i c_j a = c_j c_i a$
     \item If $k \neq i, j < \alpha$, then $d_{ij} = c_k (d_{ij} \cdot d_{jk})$
     \item If $i \neq j$, then $c_i (d_{ij} \cdot a) \cdot c_i (d_{ij} \cdot - a) = 0$
   \end{itemize}
   ${\bf CA}_{\alpha}$ is the class of all cylindric algebras of dimension $\alpha$.
\end{defin}

One may define a representation of a cylindric algebra explicitly in the following way:

\begin{defin}
 Let $\mathcal{A}$ be a cylindric algebra of dimension $\alpha$. A representation of $\mathcal{A}$ over the non-empty domain $X$ is a map $f : \mathcal{A} \hookrightarrow 2^{\prescript{\alpha}{}U}$ such that:
 \begin{enumerate}
   \item $f(1) = \bigcup \limits_{i \in I} \prescript{\alpha}{}X_i$ for some disjoint family $\{X_i\}_{i \in I}$ where each $X_i \subseteq X$
   \item $h : \mathcal{A} \to 2^{f(1)}$ is a representation of a Boolean reduct
   \item for all $\lambda, \eta < \alpha$, $x \in h(d_{\lambda \eta})$ iff $x_{\lambda} = x_{\eta}$
   \item for all $\lambda < \alpha$ and $a \in \mathcal{A}$, $x \in h(c_{\lambda}(a))$ iff there is $y \in X$ such that $x[\lambda \mapsto y] \in h(a)$
 \end{enumerate}
\end{defin}

An $\alpha$-dimensional cylindric algebra $C$ is representable, if there exists a representation of $h$.
${\bf RCA}_{\alpha}$ is the class of all representable cylindric algebras that have dimension $\alpha$. In particular, we are interested in the case $\alpha = \omega$.

It is well known that ${\bf RCA}_{\alpha}$ is a variety, ${\bf RCA}_{\alpha}$ ($\alpha \leq 2$) is finitely axiomatisable and ${\bf RCA}_{\alpha}$ ($2 < \alpha < \omega$) has no finite axiomatisation, see \cite{Henkin1988-HENCAP-4}.

Let $\mathcal{A} \in {\bf CA}_{\omega}$, then $\mathcal{A}$ has a \emph{complete representation}, if its representation preserves all existing suprema. In other words, $\mathcal{A}$ is \emph{completely representable}.

\section{${\bf RCA}_{\omega}$ and canonicity}

The following definition of an $\omega$-frame is due to \cite{Venema2013}.
\begin{defin}
  A cylindric $\omega$-frame is a structure $\mathcal{F} = \langle W, (R_i)_{i < \omega}, (E_{ij})_{i, j < \omega} \rangle$ where $(R_i)_{i < \omega}$ are binary relations and $(E_ij)_{i, j < \omega}$ are unary relations such that, for all $i, j, k < \omega$:
  \begin{enumerate}
  \item Every $R_i$ is an equivalence relation on $W$,
  \item $R_i \circ R_j = R_j \circ R_i$, that is, the set $(R_i)_{i < \omega}$ forms a commutative semigroup under composition.
  \item For all $x \in W$, $E_{ii}(x)$ holds.
  \item For all $x, y, z \in W$, $x R_i y \: \& \: E_{ij}(y) \: \& \: x R_i z \: \& \: E_{ij}(y)$ implies $y = z$.
  \item For all $x \in W$, $E_{ij}(x)$ iff there exists $y \in W$ such that $x R_k y$, $E_{ik}(y)$, and $E_{kj}(y)$.
  \end{enumerate}
  $\mathcal{C}\mathfrak{a}_{\omega}$ is the class of all $\omega$-frames.

  If $\mathcal{F} \in \mathcal{C}\mathfrak{a}_{\omega}$ and $x \in \mathcal{F}$, then $\mathcal{F}^{x}$ is a generated subframe generated by $x$, which is defined standardly. Generally, $\mathcal{F}_1$ is a generated subframe of $\mathcal{F}_2$, if $\underline{\mathcal{F}_1} \subseteq \underline{\mathcal{F}_2}$ and $\underline{\mathcal{F}_1}$ is closed under under ${R_i}_2$ equivalences for every $i < \omega$. That is:
  \begin{center}
  For all $i < \omega$ and $x \in \mathcal{F}_1$, we have ${R_i}_2(x) \subseteq \mathcal{F}_1$ and, thus, ${R_i}_1(x) = {R_i}_2(x)$.
  \end{center}
\end{defin}

We have the following connection betweens $\omega$-frames and their generated subframes, which is standard for modal logic:
\begin{prop}
Let $\mathcal{F} \in \mathcal{C}\mathfrak{a}_{\omega}$, then
\begin{enumerate}
\item $\mathcal{F} = \coprod \limits_{x \in \mathcal{F}} \mathcal{F}^{x}$,
\item $\mathfrak{Cm}(\mathcal{F}) \cong \prod \limits_{x \in \mathcal{F}} \mathfrak{Cm}(\mathcal{F}^{x})$,
\item $\mathfrak{Cm}(\mathcal{F}^{x})$ is subdirectly irreducible.
\end{enumerate}
\end{prop}

It is known that $\mathcal{C}\mathfrak{a}_{\omega}$ forms an elementary class, since one can express the conditions of an $\omega$-frame with the first-order language.

The following fact is by Venema, see \cite[Proposition 2.1.5]{Venema2013}:
\begin{prop}
An $\omega$-frame $\mathcal{F}$ is cylindric iff $\mathfrak{Cm}(\mathcal{F})$ is a cylindric algebra of dimension $\omega$.
\end{prop}
A cylindric $\omega$-frame $\mathcal{F}$ is completely representable, if $\mathfrak{Cm}(\mathcal{F})$ is completely representable as a cylindric algebra of dimension $\omega$.

We are interested in the special case of cylindric $\omega$-frames called Cartesian structure of dimension $\omega$. To be more precise:

\begin{defin}
Let $U$ be a set and $V \subseteq \prescript{\omega}{}U$ be a non-empty subset of the full Cartesian space of dimension $\omega$, then an $\alpha$-dimension Cartesian structure generated by $V$ is an $\omega$-frame $\mathfrak{S}(V) = \langle V, (R_{i})_{i < \omega}, (E_{ij})_{i, j < \omega} \rangle$ such that:
\begin{enumerate}
\item $R_i = \{ (w, v) \: | \: w, v \in V, w_k = w_k, k < \omega, i \neq k \}$
\item $E_{ij} = \{ w \in V \: | \: w_i = w_j \}$
\end{enumerate}
$\mathfrak{S}(\prescript{\omega}{}U)$ is the full $\omega$-dimensional Cartesian structure. $\mathcal{F} \mathfrak{c}\mathfrak{t}_{\omega}$ is the class of all full $\omega$-dimensional Cartesian structures.
\end{defin}

Clearly $\mathcal{F} \mathfrak{c}\mathfrak{t}_{\omega} \subseteq \mathcal{C}\mathfrak{a}_{\omega}$.

We have the following connection between ${\bf RCA}_{\omega}$, ${\bf IGs}_{\omega}$, and complex algebras of full Cartesian structures:
\begin{center}
${\bf RCA}_{\omega} = {\bf IGs}_{\omega} = {\bf S} \mathfrak{Cm} {\bf Ud} \mathcal{F} \mathfrak{c}\mathfrak{t}_{\omega} = {\bf S P} \mathfrak{Cm}  \mathcal{F} \mathfrak{c}\mathfrak{t}_{\omega}$.
\end{center}
This follows from the fact that ${\bf Cs}_{\omega} = \mathfrak{Cm} \mathcal{F} \mathfrak{c}\mathfrak{t}_{\omega}$. Every generalised cylindric set algebra is a subdirect product of cylindric set algebras, thus, a generalised cylindric set algebra is a complex algebra of disjoint union of some full Cartesian spaces. But ${\bf RCA}_{\omega}$ is the closure of ${\bf Cs}_{\omega}$ under isomorphism.

\begin{defin}
The weak Cartesian space with base $U$ and dimension $\omega$ determined by $x \in \prescript{\omega}{}U$ is the set:
\begin{center}
$\prescript{\omega}{}U^{(x)} = \{ y \in \prescript{\omega}{}U \: | \: |\{ k < \omega \: | \: x_k \neq y_k \}| < \aleph_0 \}$
\end{center}
$\mathfrak{S}(\prescript{\omega}{}U^{(x)})$ is a weak Cartesian structure of dimension $\omega$. $\mathcal{W} \mathfrak{ct}_{\omega}$ is the class of all weak Cartesian structure of dimension $\omega$ up to isomorphism.
\end{defin}
Note that we have $\mathcal{W} \mathfrak{ct}_{\omega} \subseteq \mathcal{C}\mathfrak{a}_{\omega}$.

Every cylindric set algebra is a subalgebra of some complex algebra induced by an $\omega$-dimensional Cartesian structure. In other words,

\begin{lemma}
${\bf I}{\bf Cs}_{\omega} = {\bf S} \mathfrak{Cm} \mathcal{F} \mathfrak{c}\mathfrak{t}_{\omega}$.
\end{lemma}


In this section, we reproduce the results related to characterisation ${\bf RCA}_{\omega}$. The following results are due to Goldblatt \cite{goldblatt1995elementary}. This denotes that a cylindric algebra of dimension algebra is representable iff it is isomorphic to a subalgebra of the complex algebra of disjoint sum of some full $\omega$-dimensional Cartesian structure. Assuming the duality, this is equivalent to the standard definition of representability formulated in terms of sublagebras of subdirect products.

\begin{lemma} \ref{rcachar}
${\bf RCA}_{\omega} = {\bf S} \: \mathfrak{Cm} \mathbb{S} {\bf Ud} \mathcal{F} \mathfrak{ct}_{\omega} = {\bf S} \: \mathfrak{Cm} \mathbb{S} {\bf Ud} \mathcal{W} \mathfrak{ct}_{\omega} = {\bf IGws}_{\omega}$
\end{lemma}

\begin{proof}

\end{proof}

Here we use the following fact related to canonical varieties generated by some class of complex algebras. Let ${\bf K}$ be an elementary class of relational structures, then:
\begin{center}
If ${\bf K}$ is closed under $p$-morphic images, generated subframes, and disjoint unious, \\ then ${\bf S} \mathfrak{Cm} {\bf K}$ is a canonical variety.
\end{center}
One may think of this fact a more abstract version of Fine's theorem which claims that every elementary modal logic is canonical \cite{fine1975some}. This version denotes the same fact, but it is formulated in terms of varieties BAOs generated by complex algebras of some atom structures. We provide a more precise formulation of the fact above.

Let ${\bf K}$ be a class of frames, denote the closure of ${\bf K}$ under ultraproducts as ${\bf Pu} {\bf K}$.

\begin{prop}
Let ${\bf K}$ be a class of frames, then ${\bf Pu}{\bf K} \subseteq \mathbb{H}\mathbb{S}{\bf Ud} {\bf K}$ implies that ${\bf S} \mathfrak{Cm} \mathbb{S}{\bf Ud} {\bf K}$ is a canonical variety.
\end{prop}

This is a specialised version of \cite[Theorem 4.4]{goldblatt1995elementary} formulated for dimension $\omega$.

\begin{theorem}
${\bf RCA}_{\omega}$ is a canonical variety.
\end{theorem}

\begin{proof}
We have ${\bf RCA}_{\omega} = {\bf S} \mathfrak{Cm} \mathbb{S} {\bf Ud} \mathcal{F} \mathfrak{ct}_{\omega}$.
That's enough to show that ${\bf Pu} \mathcal{F} \mathfrak{ct}_{\omega} \subseteq \mathbb{H}\mathbb{S}{\bf Ud} \mathcal{F} \mathfrak{ct}_{\omega}$.
For that, we need the following claim:
\begin{claim}\label{claim1}
${\bf Pu} \mathcal{F} \mathfrak{ct}_{\omega} \subseteq$
\end{claim}
\end{proof}

\section{Canonicity of ${\bf RCA}_n$ for finite $n$}

In this section we consider classes ${\bf RCA}_n$, where $n < \omega$ is finite.

We provide the complete proof of the following theorem \cite[Theorem 3.4.3]{hirsch2013completions}.
\begin{theorem}\label{finitecanon}
Let $\mathcal{A} \in {\bf CA}_n$, then $\mathcal{A}$ is representable iff $\mathcal{A}^{+}$ is completely representable.
\end{theorem}

For that we need such model theoretic notions as saturation and types, see \cite[Section 6.3]{hodges1993model}.

\begin{defin} Let $\mathcal{M}$ be a first-order structure of a signature $L$ and $S \subseteq \mathcal{M}$. Let $L(S)$ be an extension of $L$ with copies of elements from $S$ as additional constants. We assume that $Cnst(L)$ and $S$ are disjoint.

\begin{enumerate}
\item Let $n < \omega$, an $n$-type over $S$ is a set $\mathcal{T}$ of $L(S)$ formulas $A(\overline{x})$, where $\overline{x}$ is a fixed $n$-tuple of elements from $S$. Notation: $\mathcal{T}(\overline{x})$. A type is an $n$-type for some $n < \omega$.
\item An $n$-type $\mathcal{T}(\overline{x})$ is realised in $\mathcal{M}$, if there exists $\overline{m} \in \mathcal{M}^n$ such that $\mathcal{M} \models A(\overline{m})$ for every $A \in \mathcal{T}(\overline{x})$. $\mathcal{M}$ omits $\mathcal{T}(\overline{x})$, if $\mathcal{T}(\overline{x})$ is not realised in $\mathcal{M}$.
\item $\mathcal{T}(\overline{x})$ is finitely satisfied in $\mathcal{M}$, if every finite subtype $\mathcal{T}_0(\overline{x}) \subseteq \mathcal{T}(\overline{x})$ is realised in $\mathcal{M}$. We can reformulate that as $\mathcal{M} \models \exists \overline{a} \bigwedge \limits_{A \in \mathcal{T}_0} A(\overline{a})$.
\item Let $T$ be a theory, then a type $\mathcal{T}$ over the empty set of constants is $T$-consistent, if there exists a model $\mathcal{M} \models T$ such that $\mathcal{T}$ is finitely satisfied in $\mathcal{M}$.
\item Let $\kappa$ be a cardinal, then $\mathcal{M}$ is $\kappa$-saturated, if for every $S \subseteq \mathcal{M}$ with $|S| < \kappa$ every finitely satisfied $1$-type $\mathcal{T}$ is realised in $\mathcal{M}$.
\end{enumerate}
\end{defin}

By default, a saturated model is an $\omega$-saturated model for us.

The useful facts, they are from \cite{chang1990model} and \cite{hodges1993model}:

\begin{fact} Let $\mathcal{M}$ be an FO-structue and $\kappa$ a cardinal, then:
\begin{enumerate}
\item $\mathcal{M}$ is $\kappa$-saturated, iff every finitely satisfiable $\alpha$-type (an arbitrary $\alpha \leq \kappa$) with fewer than $\kappa$ parameters is realised in  $\mathcal{M}$.
\item If $\mathcal{M}$ is $\kappa$-saturated, then $\mathcal{M}$ is $\lambda$-saturated for every $\lambda < \kappa$.
\item \label{saturation} Every consistent theory has a $\kappa$-saturated model and every model has an elementary $\kappa$-saturated extension.
\item Let $(\mathcal{M}_i)_{i < \omega}$ a family of structures of the (at most) countable signature and $D$ a non-principal ultrafilter over $\omega$, then $\Pi_D \mathcal{M}_i$ is $\omega_1$-saturated.
\end{enumerate}
\end{fact}

\subsection{Proof of Theorem~\ref{finitecanon}}

Let $\mathcal{A} \in {\bf CA}_n$, then if $\mathcal{A}$ is completely representable, then $h$, a complete representation of $\mathcal{A}$, is atomic. That is, $(a_1, \dots, a_n) \in h(1)$, then $(a_1, \dots, a_n) \in h(y)$ for some $y \in \operatorname{At}(\mathcal{A})$.

\begin{defin} \label{theory} Let $\mathcal{A}$ be a cylindric algebra of dimension $n < \omega$.
$L(\mathcal{A})$ is the first-order language that consists of equality plus $n$-ary predicate letters $(R^n_a)_{a \in \mathcal{A}}$. The $L(\mathcal{A})$-theory $T_{\mathcal{A}}$ consists of the following sentences:
\begin{enumerate}
\item $A_+(a,b,c) := \forall x_1, \dots, x_n \: (R_a(x_1, \dots, x_n) \leftrightarrow R_b(x_1, \dots, x_n) \lor R_c(x_1, \dots, x_n))$. Informally, that means $\mathcal{A} \models a = b + c$.
\item $A_{-}(a,b) := \forall x_1, \dots, x_n \: (R_a(x_1, \dots, x_n) \leftrightarrow \neg R_b(x_1, \dots, x_n))$. That is, $\mathcal{A} \models a = - b$.
\item $A_{\neq 0}(a) := \exists x_1, \dots, x_n R_a(x_1, \dots, x_n)$. That is, $\mathcal{A} \models a \neq 0$.
\item $A_{c_i}(a) := \forall x_1, \dots, x_n (R_{c_i a}(x_1, \dots, x_n) \leftrightarrow \exists y_1, \dots y_n (R_a(y_1, \dots, y_n) \land x_i = y_j))$, for $i < n$ and $j < n$ such that $i \neq j$. Informally, $\mathcal{A} \models {c_i} a = 1$.
\item $A_{d_{ij}} := \forall x_1, \dots, x_n (R_{d_{ij}}(x_1, \dots, x_n) \leftrightarrow x_i = x_j)$, for $i, j < n$.
\end{enumerate}
\end{defin}

In fact, we need to show the following implication:
\begin{center}
If $\mathcal{A}$ is representable, then $A^{+}$ is completely representable.
\end{center}

Assume that $\mathcal{A}$ is representable, then the theory $T(\mathcal{A})$ is consistent, then it has an $\omega$-saturated model $\mathcal{M}$ by Fact~\ref{saturation}. We have the following claim:
\begin{claim}
The set $U_{x_1,\dots,x_n} = \{ a \in \mathcal{A} \: | \: \mathcal{M} \models R_a(x_1,\dots,x_n)\}$ is an ultrafilter of $\mathcal{A}$, for $x_1,\dots,x_n \in \mathcal{M}$ with $R_{1}(x_1,\dots,x_n)$.
\end{claim}
Those $U_{x_1,\dots,x_n}$'s allow us to represent atoms of $\mathcal{A}^{+}$.

We define a representation of $\mathcal{A}^{+}$ as a map $h : \mathcal{A}^{+} \to 2^{\mathcal{M}^n}$ such that:
\begin{center}
$h : S \mapsto \{ (x_1, \dots, x_n) \in 1^{\mathcal{M}} \: | \: U_{x_1,\dots,x_n} \in S \}$, for $S \in \operatorname{Spec}(\mathcal{A})$.
\end{center}

\begin{claim} Let $A_1, A_2 \in \operatorname{Spec}(\mathcal{A})$
\begin{enumerate}
\item $h(0^{\mathcal{A}^{+}}) = \emptyset$
\item $h(- A_1) = - h(A_1)$
\item $h(1^{\mathcal{A}^{+}}) = 1^{\mathcal{M}}$
\item If $S \subseteq \operatorname{Spec}(\mathcal{A})$, then $h(\bigcup S) = \bigcup \limits_{U \in S} h(U)$
\end{enumerate}
In particular, $h$ is a Boolean homomorphism.
\end{claim}

\begin{proof}
$ $

\begin{enumerate}
\item $h(0^{\mathcal{A}^{+}}) = h(\emptyset) = \emptyset$.
\item From the definiton of $h$.
\item $h(- A_1) = - h(A_1)$

Let $x_1, \dots, x_n \in 1^{\mathcal{M}}$, then we have:

\begin{center}
$(x_1, \dots, x_n) \in h(- A_1)$ iff $U_{x_1, \dots, x_n} \in - A_1$ iff $U_{x_1, \dots, x_n} \notin A_1$ iff $(x_1, \dots, x_n) \notin h(A_1)$
\end{center}
\item Let $S = \bigcup \limits_{i \in I} S_i$, where $S_i \in \operatorname{Spec}(\mathcal{A})$ for every $i \in I$.
Let $(x_1, \dots, x_n) \in 1^{\mathcal{M}}$, then we have:
\begin{center}
$(x_1, \dots, x_n) \in h(\bigcup \limits_{i \in I} S_i)$ iff $f_{x_1, \dots, x_n} \in \bigcup \limits_{i \in I} S_i$ iff $\exists i \in I \:\: f_{x_1, \dots, x_n} \in S_i$ iff $\exists i \in I \:\: (x_1, \dots, x_n) \in h(S_i)$ iff $(x_1, \dots, x_n) \in \bigcup \limits_{i \in I} S_i$
\end{center}
\end{enumerate}
\end{proof}

\begin{claim}
$h$ is injective.
\end{claim}

\begin{proof}
Let $U \in \operatorname{Spec}(\mathcal{A})$. The first is to show that $h(U)$ is non-empty. The following $n$-type:
\begin{center}
$T(x_1, \dots, x_n) = \{ R_a(x_1, \dots, x_n) \: | \: a \in U \}$
\end{center}
if finitely satisfied in $\mathcal{M}$.

Consider $T_0 = \{ R_{a_1}(x_1, \dots, x_n), \dots, R_{a_k}(x_1, \dots, x_n) \} \subseteq T$. Then $a_1, \dots, a_k \in U$ and $a = a_1 \cdot \dots \cdot a_k \in U$. By the instance of the $A_{\neq 0}(a)$-axiom, we have $\mathcal{M} \models \exists x_1, \dots, x_n R_a(x_1, \dots, x_n)$.
$a \leq a_i$ for $i \leq k$, so we have $\mathcal{M} \models \exists x_1, \dots, x_n R_{a_i}(x_1, \dots, x_n)$ for every $a_i$ with $i \leq k$ by the instance of the $A_{+}(a_i, a, a)$-axiom. That makes every finite subtype of $T$ satisfiable, thus the whole type is finitely satifiable in $\mathcal{M}$.
$\mathcal{M}$ is $\omega$-saturated, then $T$ is realised in $\mathcal{M}$ by some $(x_1, \dots, x_n) \in \mathcal{M}^n$ and, moreover, $\mathcal{M} \models 1(x_1, \dots, x_n)$. As we have already said, $U_{x_1, \dots, x_n}$ is an ultrafilter, but $U_{x_1, \dots, x_n} \subseteq U$, thus $U = U_{x_1, \dots, x_n}$, so $(x_1, \dots, x_n) \in h(U)$.

That makes $h$ one-to-one.
\end{proof}

\begin{claim}
$ $

\begin{enumerate}
\item $h(c_i{^{\mathcal{A}^{+}}} U) = C_i (h(U))$
\item $h(d_{ij}^{\mathcal{A}^{+}}) = D_{ij} \subseteq \operatorname{Spec}(\mathcal{A})$
\end{enumerate}
\end{claim}

\begin{proof}
$ $

\begin{enumerate}
\item Let $\overline{x} = (x_1, \dots, x_n) \in \mathcal{M}^{n}$ and $S \subseteq \operatorname{Spec}(\mathcal{A})$. Assume $(x_1, \dots, x_n) \in h(c_i{^{\mathcal{A}^{+}}} S)$.

Let us show that $\overline{x} \in C_i (h(S))$, that is, there exists $\overline{y} = (y_1, \dots, y_n) \in h(S)$ such that $\overline{x} \equiv_{i} \overline{y}$.

Then $\mathcal{M} \models 1(x_1, \dots, x_n)$ and $U_{x_1, \dots x_n} \in c_i{^{\mathcal{A}^{+}}} S$. But $\mathcal{A}^{+}$ is the complex algebra of the ultrafilter frame $\mathcal{F}_{\mathcal{A}}$.
Then we have:
\begin{center}
$c_i{^{\mathcal{A}^{+}}} S = \{ U_1 \in \operatorname{Spec}(\mathcal{A}) \: | \: \exists U' \in S  \: U_1 R_{i} U' \}$
\end{center}
Then there must be an ultrafilter $U' \in S$ such that $U_{x_1, \dots x_n} R_{i} U'$, that is, $c_i a \in U_{x_1, \dots x_n}$ whenever $a \in U'$.
Hence $\mathcal{M} \models R_{c_i}(x_1, \dots x_n)$. By the $A_{c_i}(a)$-axiom, we have
\begin{center}
$\mathcal{M} \models \exists z_1, \dots, z_n (R_a(z_1, \dots, z_n) \land x_i = z_j)$ for $i < n$ and $j < n$ such that $i \neq j$.
\end{center}
Consider the following $n$-type with free variables $z_1, \dots, z_n$ and parameters $x_1, \dots, x_n \in \mathcal{M}$:
\begin{center}
$T(z_1, \dots, z_n) = \{ R_a(z_1, \dots, z_n) \land x_i = z_j \: | \: i < n, j < n, i \neq j, a \in U' \}$.
\end{center}
Let us show that $T(z_1, \dots, z_n)$ is finitely satifsiable in $\mathcal{M}$.
Consider a finite subset of $T$, say $T_0 = \{ R_{b_k}(z_1, \dots, z_n) \land x_i = y_j \: | \: i < n, j < n, i \neq j, b_k \in U', k < \omega \}$.
We put $p = p_1 \cdot \dots \cdot p_k$ and $p \in U'$ since $U'$ is a filter. Then we have:
\begin{center}
$\mathcal{M} \models \exists z_1, \dots, z_n (R_b(z_1, \dots, z_n) \land x_i = z_j)$ for $i < n$ and $j < n$ such that $i \neq j$
\end{center}
Thus, we have, as required:
\begin{center}
$\mathcal{M} \models \exists z_1, \dots, z_n \bigwedge \limits_{i = 1}^{k} (R_{b_k}(z_1, \dots, z_n) \land x_i = z_j)$ for $i < n$ and $j < n$ such that $i \neq j$.
\end{center}
As above, using $\omega$-saturation, we conclude that $T$ is realised in $\mathcal{M}$ at an $n$-tuple $(y_1, \dots, y_n) = \overline{y}$.
Then we have:
\begin{center}
$\mathcal{M} \models 1(\overline{y})$, $\overline{x} \equiv_i \overline{y}$, $U_{\overline{y}} \supseteq U'$
\end{center}
Then $U_{\overline{y}} = U'$, then $\overline{y} \in h(S)$. Then $\overline{x} \in C_i (h (S))$.

Suppose for the converse, $\overline{x} = (x_1, \dots, x_n) \in C_i (h(S))$. We need $\overline{x} \in h(c_i(S))$.
Then there exists $\overline{y} = (y_1, \dots, y_n)$ such that $\overline{x} \equiv_i \overline{y}$ and $\overline{y} \in h(S)$. Then there exists an ultrafilter $U_{y_1, \dots, y_n} \in S$. Let us show that $\mathcal{M} \models 1(x_1, \dots, x_n)$ and $U_{x_1, \dots, x_n} \in c_i U_{y_1, \dots, y_n}$.
Let $a \in U_{y_1, \dots, y_n}$. Then we have $\mathcal{M} \models R_a(y_1, \models, y_n)$. By the $A_{c_i}(a)$ axiom, we have $\mathcal{M} \models R_{c_i a}(x_1, \dots, x_n)$. Then $\mathcal{M} \models 1(x_1, \dots, x_n)$ and $c_i a \in U_{x_1, \dots, x_n}$, thus, $\overline{x} \in h(c_i(S))$.
\item Let us show that $h$ preserves cylindrifications.

Let $(x_1, \dots, x_n) \in \mathcal{M}^n$. Then $(x_1, \dots, x_n) \in D_{ij}$ iff $\mathcal{M} \models 1(x_1, \dots, x_n)$ and $x_i = x_j$ iff $U_{x_1, \dots, x_n} \in d_{ij}^{\mathcal{A}^{+}} = \{ U \in \operatorname{Spec}(\mathcal{A}) \: | \: d_ij \in U \}$ iff $\mathcal{M} \models d_{ij}^{\mathcal{M}}(x_1, \dots, x_n)$.
\end{enumerate}
\end{proof}

\section{Representability games}

\bibliographystyle{plain}
\bibliography{Text}

\end{document}

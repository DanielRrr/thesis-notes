\documentclass[a4paper]{article}
\usepackage{amsmath}
\usepackage{amsthm}
\usepackage{amsfonts}
\usepackage{amssymb}
\usepackage{bussproofs}
\usepackage{mathtools}
\usepackage{verbatim}
\usepackage{dsfont}
\usepackage{mathabx}
\usepackage[all, 2cell]{xy}
\usepackage[all]{xy}
\usepackage{wasysym}
\usepackage{rotating}
\usepackage{geometry}
\usepackage{trfsigns}
\usepackage{cmll}
\usepackage{authblk}
\usepackage{hyperref}
\usepackage{cleveref}
\usepackage{lipsum}
\usepackage{extpfeil}
\usepackage{soul}
\usepackage{graphicx}

\newcommand\mapsfrom{\mathrel{\reflectbox{\ensuremath{\mapsto}}}}

\theoremstyle{defin}
\newtheorem{definition}{Definition}

\theoremstyle{theorem}
\newtheorem{theorem}{Theorem}

\theoremstyle{claim}
\newtheorem{claim}{Claim}

\theoremstyle{prop}
\newtheorem{prop}{Proposition}

\theoremstyle{lemma}
\newtheorem{lemma}{Lemma}

\theoremstyle{fact}
\newtheorem{fact}{Fact}

\theoremstyle{ex}
\newtheorem{ex}{Example}


\theoremstyle{col}
\newtheorem{col}{Corollary}

\let\strokeL\L
\DeclareRobustCommand{\L}{\ifmmode\mathbf{L}\else\strokeL\fi}

\author{Daniel Rogozin}
\date{}
\title{Characterising representable positive relation algebras via Priestley duality}

\begin{document}
\maketitle

\nocite{*}

\section{Distributive lattice representation and Priestley duality}

\begin{lemma}
Let $h : \mathcal{L} \to \mathcal{R}$ be a representation, then then
\begin{center}
$h^{-1}[x] = \{ a \in \mathcal{L} \: | \: x \in h(a) \}$
\end{center}
is prime filter in $\mathcal{L}$.
\end{lemma}

\begin{proof}
Let $c \in h^{-1}[x]$ and $c \leq d$, then $x \in h(c)$, but $h$ is order-preserving, so $x \in h(d)$. If $c, d \in h^{-1}[x]$, then $x \in h(c)$ and $x \in h(d)$, so $x \in h(c) \cap h(d)$, then $x \in h(c \cdot d)$, so $c \cdot d \in h^{-1}[x]$. Let $c + d \in h^{-1}[x]$, then $x \in h(c + d) = h(c) \cup h(d)$, so either $x \in h(c)$ or $x \in h(d)$, so either $c \in h^{-1}[x]$ or $d \in h^{-1}[x]$.
\end{proof}

\subsection{Priestley duality}


\section{Representatiting positive relation algebras}

\section{Spectral spaces for positive relation algebras}

\section{Complete representability}

\subsection{Completely representable distibutive lattices}

TODO: read \cite{egrot2012completely}

\subsection{Completely representable positive relation algebras}

\section{The main result}

For that we need such model theoretic notions as saturation and types, see \cite[Section 6.3]{hodges1993model}.

\begin{definition} Let $\mathcal{M}$ be a first-order structure of a signature $L$ and $S \subseteq \mathcal{M}$. Let $L(S)$ be an extension of $L$ with copies of elements from $S$ as additional constants. We assume that $Cnst(L)$ and $S$ are disjoint.

\begin{enumerate}
\item Let $n < \omega$, an $n$-type over $S$ is a set $\mathcal{T}$ of $L(S)$ formulas $A(\overline{x})$, where $\overline{x}$ is a fixed $n$-tuple of elements from $S$. Notation: $\mathcal{T}(\overline{x})$. A type is an $n$-type for some $n < \omega$.
\item An $n$-type $\mathcal{T}(\overline{x})$ is realised in $\mathcal{M}$, if there exists $\overline{m} \in \mathcal{M}^n$ such that $\mathcal{M} \models A(\overline{m})$ for every $A \in \mathcal{T}(\overline{x})$. $\mathcal{M}$ omits $\mathcal{T}(\overline{x})$, if $\mathcal{T}(\overline{x})$ is not realised in $\mathcal{M}$.
\item $\mathcal{T}(\overline{x})$ is finitely satisfied in $\mathcal{M}$, if every finite subtype $\mathcal{T}_0(\overline{x}) \subseteq \mathcal{T}(\overline{x})$ is realised in $\mathcal{M}$. We can reformulate that as $\mathcal{M} \models \exists \overline{a} \bigwedge \limits_{A \in \mathcal{T}_0} A(\overline{a})$.
\item Let $T$ be a theory, then a type $\mathcal{T}$ over the empty set of constants is $T$-consistent, if there exists a model $\mathcal{M} \models T$ such that $\mathcal{T}$ is finitely satisfied in $\mathcal{M}$.
\item Let $\kappa$ be a cardinal, then $\mathcal{M}$ is $\kappa$-saturated, if for every $S \subseteq \mathcal{M}$ with $|S| < \kappa$ every finitely satisfied $1$-type $\mathcal{T}$ is realised in $\mathcal{M}$.
\end{enumerate}
\end{definition}

By default, a saturated model is an $\omega$-saturated model for us.

The useful facts, they are from \cite{chang1990model} and \cite{hodges1993model}:

\begin{fact} Let $\mathcal{M}$ be an FO-structue and $\kappa$ a cardinal, then:
\begin{enumerate}
\item $\mathcal{M}$ is $\kappa$-saturated iff every finitely satisfiable $\alpha$-type (an arbitrary $\alpha \leq \kappa$) with fewer than $\kappa$ parameters is realised in  $\mathcal{M}$.
\item If $\mathcal{M}$ is $\kappa$-saturated, then $\mathcal{M}$ is $\lambda$-saturated for every $\lambda < \kappa$.
\item \label{saturation} Every consistent theory has a $\kappa$-saturated model and every model has an elementary $\kappa$-saturated extension.
\item Let $(\mathcal{M}_i)_{i < \omega}$ a family of structures of the (at most) countable signature and $D$ a non-principal ultrafilter over $\omega$, then $\Pi_D \mathcal{M}_i$ is $\omega_1$-saturated.
\end{enumerate}
\end{fact}

Let $\mathcal{A}$ be a positive relation algebra, define the first-order relational language of the form
\begin{center}
$\mathcal{L}(\mathcal{A}) = (=, \{ R^2_a \}_{a \in \mathcal{A}})$
\end{center}

The $\mathcal{L}(\mathcal{A})$-theory ${T}_{\mathcal{A}}$ consists of the following statements:
\begin{itemize}
\item $\sigma_1 = \forall x \forall y ({\bf 1}'(x, y) \leftrightarrow (x = y))$
\item $\sigma_+(R, S, T) = \forall x \forall y (R(x, y) \leftrightarrow S(x, y) \lor T (x, y))$
\item $\sigma_{\cdot}(R, S, T) = \forall x \forall y (R(x, y) \leftrightarrow S(x, y) \land T (x, y))$
\item $\sigma_{;}(R, S, T) = \forall x \forall y (R(x, y) \leftrightarrow \exists z (S(x, z) \land T(z, y)))$
\item $\sigma_{\smile}(R, S) = \forall x \forall y (R(x, y) \leftrightarrow S(y, x))$
\item $\sigma_{\neq 0} = \exists x \exists y R(x, y)$ for any $R_a$ such that $a \neq 0$
\item $\sigma_{0} = \neg \exists x \exists y 0(x, y)$
\item $\sigma_{{\bf 1}} = \forall x \forall y (R(x, y) \rightarrow {\bf 1}(x, y))$
\end{itemize}

\begin{prop}
$T_{\mathcal{A}}$ is satisfiable whenever $\mathcal{A}$ is representable.
\end{prop}

\begin{theorem}
Let $\mathcal{A}$ be a positive relation algebra, then $\mathcal{R}$ is representable iff $(\mathcal{R}_{+})^{+}$ is completely representable.
\end{theorem}

\begin{theorem}
${\bf RPRA}$ is a canonical variety.
\end{theorem}

\bibliographystyle{alpha}
\bibliography{Text}
\end{document}

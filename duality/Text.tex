\documentclass[a4paper]{article}
\usepackage{amsmath}
\usepackage{amsthm}
\usepackage{amsfonts}
\usepackage{amssymb}
\usepackage{bussproofs}
\usepackage{mathtools}
\usepackage{verbatim}
\usepackage{dsfont}
\usepackage{mathabx}
\usepackage[all, 2cell]{xy}
\usepackage[all]{xy}
\usepackage{wasysym}
\usepackage{rotating}
\usepackage{geometry}
\usepackage{trfsigns}
\usepackage{cmll}
\usepackage{authblk}
\usepackage{hyperref}
\usepackage{cleveref}
\usepackage{lipsum}
\usepackage{extpfeil}
\usepackage{soul}
\usepackage{graphicx}

\newcommand\mapsfrom{\mathrel{\reflectbox{\ensuremath{\mapsto}}}}

\theoremstyle{defin}
\newtheorem{definition}{Definition}

\theoremstyle{theorem}
\newtheorem{theorem}{Theorem}

\theoremstyle{claim}
\newtheorem{claim}{Claim}

\theoremstyle{prop}
\newtheorem{prop}{Proposition}

\theoremstyle{lemma}
\newtheorem{lemma}{Lemma}

\theoremstyle{fact}
\newtheorem{fact}{Fact}

\theoremstyle{ex}
\newtheorem{ex}{Example}


\theoremstyle{col}
\newtheorem{col}{Corollary}

\let\strokeL\L
\DeclareRobustCommand{\L}{\ifmmode\mathbf{L}\else\strokeL\fi}

\author{Daniel Rogozin}
\date{}
\title{Characterising representable positive relation algebras via Priestley duality}

\begin{document}
\maketitle

\nocite{*}

\section{Distributive lattice representation and priestley duality}

\begin{lemma}
Let $h : \mathcal{L} \to \mathcal{R}$ be a representation, then then
\begin{center}
$h^{-1}[x] = \{ a \in \mathcal{L} \: | \: x \in h(a) \}$
\end{center}
is prime filter in $\mathcal{L}$.
\end{lemma}

\begin{proof}
Let $c \in h^{-1}[x]$ and $c \leq d$, then $x \in h(c)$, but $h$ is order-preserving, so $x \in h(d)$. If $c, d \in h^{-1}[x]$, then $x \in h(c)$ and $x \in h(d)$, so $x \in h(c) \cap h(d)$, then $x \in h(c \cdot d)$, so $c \cdot d \in h^{-1}[x]$. Let $c + d \in h^{-1}[x]$, then $x \in h(c + d) = h(c) \cup h(d)$, so either $x \in h(c)$ or $x \in h(d)$, so either $c \in h^{-1}[x]$ or $d \in h^{-1}[x]$.
\end{proof}

\section{Positive relation algebras and their representatition}

\section{Spectral spaces for positive relation algebras}

\section{Complete representability}

An element $a$ of a lattice $\mathcal{L}$ is \emph{completely join-irreducible} if for every $A \subseteq \mathcal{L}$ such that $a \leq \bigvee A$ there exists $b \in A$ such that $a \leq b$.

A distributive lattice $\mathcal{L}$ is \emph{completely representable} if there exists a representation $h : \mathcal{L} \to \mathcal{R}$ such that
\begin{center}
$f(\Sigma A) = \bigcup \{ f(a) \: | \: a \in A\}$

$f(\Pi A) = \bigcap \{ f(a) \: | \: a \in A \}$
\end{center}
for each $A \subseteq \mathcal{L}$ such that $\Sigma A$ and $\Pi A$ exist.

A representation $h : \mathcal{L} \to \mathcal{R}$ is \emph{refined} if for all $a \in \mathcal{L}$ there exists a completely join-irreducible $j$ such that $a \in h(j)$.

\begin{lemma}
Let $h : \mathcal{L} \to \mathcal{R}$ be a \emph{refined} representation, then each $h^{-1}[x]$ is principal and is generated by completely join-irreducible element.
\end{lemma}

\begin{theorem}
Let $\mathcal{L}$ be a distributive lattice and let $h : \mathcal{L} \to \mathcal{R}$ be representation of $\mathcal{L}$ over the base set $X$, then $h$ is refined iff it is complete.
\end{theorem}

\begin{proof}
Suppose $h$ is refined. Take any $b \in \mathcal{L}$, then
\begin{center}
$h(b) = \bigcup \{ h(j) \: | \: j \in \mathcal{JR}(\mathcal{L}) \: \& \: j \leq a \}$
\end{center}

Take $A \subseteq \mathcal{L}$ such that $\Sigma A$ and $\Pi A$ exist.

TODO:

For converse, suppose $h$ is complete
\end{proof}

\subsection{Completely representable distibutive lattices}

\subsection{Completely representable positive relation algebras}

\section{The main result}

\begin{theorem}
Let $\mathcal{A}$ be a positive relation algebra, then $\mathcal{R}$ is representable iff $(\mathcal{R}_{+})^{+}$ is completely representable.
\end{theorem}

\begin{theorem}
${\bf RPRA}$ is a canonical variety.
\end{theorem}

\end{document}

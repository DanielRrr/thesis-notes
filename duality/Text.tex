\documentclass[a4paper]{article}
\usepackage{amsmath}
\usepackage{amsthm}
\usepackage{amsfonts}
\usepackage{amssymb}
\usepackage{bussproofs}
\usepackage{mathtools}
\usepackage{verbatim}
\usepackage{dsfont}
\usepackage{mathabx}
\usepackage[all, 2cell]{xy}
\usepackage[all]{xy}
\usepackage{wasysym}
\usepackage{rotating}
\usepackage{geometry}
\usepackage{trfsigns}
\usepackage{cmll}
\usepackage{authblk}
\usepackage{hyperref}
\usepackage{cleveref}
\usepackage{lipsum}
\usepackage{extpfeil}
\usepackage{soul}
\usepackage{graphicx}

\newcommand\mapsfrom{\mathrel{\reflectbox{\ensuremath{\mapsto}}}}

\theoremstyle{defin}
\newtheorem{definition}{Definition}

\theoremstyle{theorem}
\newtheorem{theorem}{Theorem}

\theoremstyle{claim}
\newtheorem{claim}{Claim}

\theoremstyle{prop}
\newtheorem{prop}{Proposition}

\theoremstyle{lemma}
\newtheorem{lemma}{Lemma}

\theoremstyle{fact}
\newtheorem{fact}{Fact}

\theoremstyle{ex}
\newtheorem{ex}{Example}


\theoremstyle{col}
\newtheorem{col}{Corollary}

\let\strokeL\L
\DeclareRobustCommand{\L}{\ifmmode\mathbf{L}\else\strokeL\fi}

\author{Daniel Rogozin}
\date{}
\title{Characterising representable positive relation algebras via Priestley duality}

\begin{document}
\maketitle

\nocite{*}

\section{Priestley duality}

\section{Positive relation algebras and their representatition}

\section{Spectral spaces for positive relation algebras}

\section{Complete representability}

\subsection{Completely representable distibutive lattices}

\subsection{Completely representable positive relation algebras}

\section{The main result}

\begin{theorem}
Let $\mathcal{R}$ be a positive relation algebra, then $\mathcal{R}$ is representable iff $(\mathcal{R}_{+})^{+}$ is completely representable.
\end{theorem}

\begin{theorem}
${\bf RPRA}$ is a canonical variety.
\end{theorem}

\end{document}

\documentclass[a4paper]{article}
\usepackage{amsmath}
\usepackage{amsthm}
\usepackage{amsfonts}
\usepackage{amssymb}
\usepackage{bussproofs}
\usepackage{mathtools}
\usepackage{verbatim}
\usepackage{dsfont}
\usepackage{mathabx}
\usepackage[all, 2cell]{xy}
\usepackage[all]{xy}
\usepackage{wasysym}
\usepackage{rotating}
\usepackage{geometry}
\usepackage{trfsigns}
\usepackage{cmll}
\usepackage{authblk}
\usepackage{hyperref}
\usepackage{cleveref}
\usepackage{lipsum}
\usepackage{extpfeil}
\usepackage{soul}
\usepackage{graphicx}

\newcommand\mapsfrom{\mathrel{\reflectbox{\ensuremath{\mapsto}}}}

\theoremstyle{defin}
\newtheorem{defin}{Definition}

\theoremstyle{theorem}
\newtheorem{theorem}{Theorem}

\theoremstyle{prop}
\newtheorem{prop}{Proposition}

\theoremstyle{lemma}
\newtheorem{lemma}{Lemma}

\theoremstyle{fact}
\newtheorem{fact}{Fact}

\theoremstyle{ex}
\newtheorem{ex}{Example}


\theoremstyle{col}
\newtheorem{col}{Corollary}

\let\strokeL\L
\DeclareRobustCommand{\L}{\ifmmode\mathbf{L}\else\strokeL\fi}

\author{Daniel Rogozin}
\date{}
\title{Representable cylindric algebras of dimension $\omega$: the aspects of canonicity and axiomatisability}

\begin{document}

\maketitle

\section{Intro}

\section{The problem itself}

Suppose $\mathcal{C} \in {\bf RCA}_{\omega}$, whether $\mathcal{C}^{+}$ has a complete, $\omega$-dimensional representation? \cite{hirsch2002relation}

\section{Boolean algebras with operators and cylindric algebras, a bit of the backgroud}

Let $a \in \mathcal{B}$ be an element of a Boolean algebra $\mathcal{B}$, $a$ is called an atom, if for every $b \in \mathcal{B}$
$b < a$ implies $b = 0$. That is, an atom is a minimal non-zero element. $\operatorname{At}(\mathcal{B})$ is the set of all atoms
of $\mathcal{B}$.

Let $\mathcal{B}$ be a Boolean algebra and $\mathcal{F}$ a field of sets such that $h : \mathcal{B} \to \mathcal{F}$ is a
representation of $\mathcal{B}$, then $\mathcal{B}$ is a complete representation of $\mathcal{B}$, if for every
$A \subseteq \mathcal{B}$ whenever $\Sigma A$ we have the following:
\begin{center}
  $h(\Sigma A) = \bigcup h[A]$
\end{center}

\begin{theorem} \label{completeboolean}
  Let $\mathcal{B}$ be a Boolean algebra, then $\mathcal{B}$ is atomic iff $\mathcal{B}$ is completely representable. See \cite[Corollary 6]{hirsch1997complete}.
\end{theorem}

\begin{defin}

$ $

  \begin{enumerate}
    \item Let $\mathcal{B} = \langle B, +, -, 0, 1 \rangle$ be a Boolean algebra. An operator is an $n$-ary function $\Omega : B^n \to B$ satisfying the following conditions:
    \begin{itemize}
      \item Normality: for all $b_0, \dots, b_{n - 1} \in B$, if $b_1 = 0$ for some $i < n$, then

\begin{center}
      $\Omega(b_0, \dots, b_{i - 1}, 0, b_{i+1}, \dots, b_{n - 1}) = 0$
\end{center}
      \item Additivity: for all $b_0, \dots, b_{n - 1}, b, b' \in B$ we have
      \begin{center}
        $\Omega(b_0, \dots, b_{i - 1}, (b + b'), b_{i+1}, \dots, b_{n - 1}) = \Omega(b_0, \dots, b_{i - 1}, b, b_{i+1}, \dots, b_{n - 1}) + \Omega(b_0, \dots, b_{i - 1}, b', b_{i+1}, \dots, b_{n - 1})$
      \end{center}
    \end{itemize}
    \item Let $I$ be an index set, a Boolean algebra with operators (BAO) is an algebra $\langle B, +, -, 0, 1, \{ \Omega_{i} \}_{i \in I} \rangle$ such that $\langle B, +, -, 0, 1 \rangle$ is a Boolean algebra and for each $i \in I$ $\Omega_{i}$ is an operator.
  \end{enumerate}
\end{defin}

\begin{defin} Let $\mathcal{B} = \langle B, +, -, 0, 1, \{ \Omega_{i} \}_{i \in I} \rangle$ be a BAO, then

  \begin{enumerate}
    \item An opetator $\Omega$ is completely additive, if for each $b_0, \dots, b_{n - 1} \in B$ and $X \subseteq B$, one has

    \begin{center}
      $\Omega(b_0, \dots, b_{i-1}, \sum X, b_{i+1}, \dots, b_{n - 1}) = \sum \limits_{x \in X} \Omega(b_0, \dots, b_{i-1}, x, b_{i+1}, \dots, b_{n - 1})$
    \end{center}
    \item $\mathcal{B}$ is completely additive, if for each $i \in I$ $\Omega_{i}$ is additive,
    \item A class $\mathcal{K}$ of BAOs is completely additive, if every $\mathcal{B} \in \mathcal{K}$ is completely additive.
  \end{enumerate}
\end{defin}

\subsection{Atom structures and canonical extensions}

\begin{defin}  Let $I$ be an index set and $\{ \Omega_i\}_{i \in I}$ a set of function symbols
\begin{enumerate}
  \item An atom structure is a relational structrure
  $\mathcal{F} = \langle W, \{ R_{i} \}_{i \in I} \rangle$
  such that $R_{i}$ is a $n+1$-ary relation symbol, where $\Omega_{i \in I}$ is an $n$-ary function symbol,
  \item Let $\mathcal{B}$ be an atomic BAO of the signature $I$,
the atom structure of $\mathcal{B}$, written as $\mathfrak{At} \mathcal{B}$, is an atom structure $\langle \operatorname{At}(\mathcal{B}), \{ R_{i}\}_{i \in I} \rangle$ such that for each
$a, b_0, \dots, b_{n + 1} \in \operatorname{At}(\mathcal{B})$ and for each $i \in I$
\begin{center}
  $\mathfrak{At} \mathcal{B} \models R_{i}(a,  b_0, \dots, b_{n + 1})$ iff $\mathcal{B} \models a \leq \Omega_{i}(b_0, \dots, b_{n + 1})$
\end{center}
\item Let $\mathcal{F} = \langle W, \{ R_{i} \}_{i \in I} \rangle$ be an atom structure, the complex algebra of $\mathcal{F}$, written as $\bf{Cm} \mathcal{F}$, is a BAO
$\langle \mathcal{P}(W), \cup, -, \emptyset, W, \{\Omega_{R_{i}}\}_{i \in I} \rangle$ such that
for all $X_0, \dots, X_{n - 1} \subseteq W$ and for each $i \in I$
\begin{center}
  $\Omega_{R_{i}}(X_0, \dots, X_{n - 1}) = \{ a \in W \: | \: \exists b_0 \in X_0 \dots \exists b_{n -1} \in X_{n - 1} \: \mathcal{F} \models R_{i}(a, b_0, \dots, b_{n - 1})\}$
\end{center}
\end{enumerate}
\end{defin}

The following duality is due to Thomason \cite{thomason1975categories}.

\begin{fact}
  $ $

  \begin{enumerate}
    \item Let $\mathcal{B}$ be a complete atomic BAO, then $\mathcal{B} \cong \mathfrak{Cm} (\bf{At}(\mathcal{B}))$,
    \item Let $\mathcal{F}$ be an atom structure, then $\mathcal{F} \cong \mathfrak{At}(\mathfrak{Cm}(\mathcal{B}))$.
  \end{enumerate}
\end{fact}

Let $A$ be a non-empty subset of a Boolean algebra $\mathcal{B}$, $A$ is a \emph{filter}, if $A$ is closed under finite infima and it is upward closed. $A$ is an ultrafilter, if it has no non-trivial extensions. That is, if $A \subseteq A'$, then $A' = \mathcal{B}$. This is a well-known fact that every filter can be extended to a maximal one using Zorn's lemma.

The following definition is due to, for example, \cite[Definition 5.40]{venema2010}.

\begin{defin}
  Let $\mathcal{B} = \langle B, +, -, 0, 1, \{ \Omega_i \}_{i \in I} \rangle$ be a BAO and ${\bf Uf}(\mathcal{B})$ the set of its ultrafilters. The ultrafilter frame of $\mathcal{B}$ (or canonical frame) is a relational structure $\mathcal{F}_{\mathcal{B}} = \langle {\bf Uf}(\mathcal{B}), R_{\Omega_i} \rangle$ such that for each ultrafilters $\beta_0, \dots, \beta_{n - 1}, \gamma$ one has
  \begin{center}
    ${\bf Uf}(\mathcal{B}) \models R_{\Omega_i}(\beta_0, \dots, \beta_{n - 1}, \gamma)$ iff $\{ \Omega(b_0, \dots, b_{n - 1}) \: | \: b_0 \in \beta_0, \dots, b_{n - 1} \in \beta_{n - 1}\} \subseteq \gamma$.
  \end{center}
\end{defin}

\begin{defin} Let $\mathcal{B}$ be a BAO, then

  \begin{enumerate}
    \item The canonical extension of $\mathcal{B}$ is a complex algebra of the canonical frame $\mathfrak{Cm}(\mathcal{F}_{\mathcal{B}})$ denoted as $\mathcal{B}^{+}$,
    \item The class of BAOs is canonical, if it is closed under canonical extensions.
  \end{enumerate}
\end{defin}

\begin{theorem} Let $\mathcal{A}$, $\mathcal{B}$ be BAOs,

\begin{enumerate}
  \item There exists $\iota : \mathcal{A} \hookrightarrow \mathcal{A}^{+}$ such that
  $\iota : a \mapsto \{ \gamma \in {\bf Uf}(\mathcal{A}) \: | \: a \in \gamma \}$.
  \item $i : \mathcal{A} \hookrightarrow \mathcal{B}$ implies
  $i^{+} : \mathcal{A}^{+} \hookrightarrow \mathcal{B}^{+}$
\end{enumerate}
\end{theorem}

\begin{fact}
\end{fact}

\section{Cylindric algebras and cylindric set algebras}

 Let $\alpha$ be an ordinal. Let $\prescript{\alpha}{}U$ be the set of all functions mapping $\alpha$ to a non-empty set $U$. We denote $x(i) = x_i$ for
$x \in \prescript{\alpha}{}U$ and $i < \alpha$.

A subset of $\prescript{\alpha}{}U$ is an $\alpha$-ry relation on $U$. For $i, j < \alpha$, the \emph{$i,j$-diagonal} $D_{ij}$ is the set of all elements of $\prescript{\alpha}{}U$ such that $y_i = y_j$.

If $i < \alpha$ and $X$ is an $\alpha$-ry relation on $U$, then
the $i$-th cylindrification $C_i X$ is the set of all elements of $U$ that agree with some element of $X$ on each coordinate except, perhaps, the $i$-th one. To be more precise,
\begin{center}
$C_i X = \{ y \in \prescript{\alpha}{}U \: |
\: \exists x \in X \forall i < \alpha \: (i \neq j \Rightarrow y_j = x_j)\}$.
\end{center}
We define the following equivalence relation for $i < \alpha$ and $x, y \in \prescript{\alpha}{}U$:
\begin{center}
  $x \equiv_i y \Leftrightarrow \forall j \in \alpha \: (i \neq j \Rightarrow x(i) = y(j))$
\end{center}
Then one may reformulate the definition of the $i$-th cylindrification in the following way:

\begin{center}
  $C_i X = \{ y \in \prescript{\alpha}{}U \: | \: \exists x \in X \:\: x \equiv_i y \}$
\end{center}

According to this version of the definiton, one may think of the cylindrification as an ${\bf S}5$ modal operator.

The following definition is due to, e.g., \cite{sankappanavar1981course}:

\begin{defin}
  Let $(\mathcal{A}_i)_{i \in I}$ be a family of algebras (of an abstract signature) and $\mathcal{A}$ is a subalgebra of $\prod \limits_{i \in I} \mathcal{A}_i$, then $\mathcal{A}$ is a subdirect product, if every projection is onto. That is, for every $i \in I$, $\pi_i[\mathcal{A}] = \mathcal{A}_i$.
\end{defin}

\begin{defin}
  A cylindic set algebra of dimension $\alpha$ is an algebra consisting of a set $S$ of $\alpha$-ry relation on some base set $U$
    with the constants and operations $0 = \emptyset$, $1 = \prescript{\alpha}{}U$, $\cap$, $-$, the diagonal elements $\{ D_{ij} \}_{i, j < \alpha}$, the cylindrifications $\{ C_i \}_{i < \alpha}$. A generalised cylindric set algebra of dimension $\alpha$ is a subdirect of cylindric algebras that have dimension $\alpha$.
\end{defin}

\begin{defin}
    A cylindric algebra of dimension $\alpha$ is an algebra $\mathcal{C} = \langle \mathcal{B}, \{ c_i \}_{i < \alpha}, \{ d_{ij} \}_{i, j < \alpha} \rangle$ such that
    \begin{itemize}
      \item $\mathcal{B}$ is a Boolean algebra, for each $i, j < \alpha$ $c_i$ is an operator and $d_{ij} \in \mathcal{B}$
      \item For each $i < \alpha$, $a \leq c_i a$, $c_i (a \land c_i b) = c_i a \land c_i b$ and $d_{ii} = 1$
      \item For every $i, j < \alpha$, $c_i c_j a = c_j c_i a$
      \item If $k \neq i, j < \alpha$, then $d_{ij} = c_k (d_{ij} \land d_{jk})$
      \item If $i \neq j$, then $c_i (d_{ij} \land a) \land c_i (d_{ij} \land - a) = 0$
    \end{itemize}
    ${\bf CA}_{\alpha}$ is the class of all cylindric algebras of dimension $\alpha$.
\end{defin}

\begin{defin}
  Let $\mathcal{A} = \langle A, +, -, 0, 1, (c_i)_{i < \alpha}, (d_ij)_{i,j < \alpha} \rangle \in {\bf CA}_{\alpha}$ and $\beta < \alpha$, then the $\beta$-reduct of $\mathcal{A}$ is an algebra $\mathcal{A}_{\beta} = \langle A, +, -, 0, 1, (c_i)_{i < \beta}, (d_ij)_{i,j < \beta} \rangle$,
\end{defin}

One may define a representation of a cylindric algebra explicitly in the following way:

\begin{defin}
  Let $\mathcal{A}$ be a cylindric algebra of dimension $\alpha$. A representation of $\mathcal{A}$ over the non-empty domain $X$ is a map $f : \mathcal{A} \hookrightarrow 2^{\prescript{\alpha}{}U}$ such that:
  \begin{enumerate}
    \item $f(1) = \bigcup \limits_{i \in I} \prescript{\alpha}{}X_i$ for some disjoint family $\{X_i\}_{i \in I}$ where each $X_i \subseteq X$
    \item $h : \mathcal{A} \to 2^{f(1)}$ is a representation of a Boolean reduct
    \item for all $\lambda, \eta < \alpha$, $x \in h(d_{\lambda \eta})$ iff $x_{\lambda} = x_{\eta}$
    \item for all $\lambda < \alpha$ and $a \in \mathcal{A}$, $x \in h(c_{\lambda}(a))$ iff there is $y \in X$ such that $x[\lambda \mapsto y] \in h(a)$
  \end{enumerate}
\end{defin}

An $\alpha$-dimensional cylindric algebra $C$ is representable, if it is isomorphic to a generalised cylindric set algebra
of dimension $\alpha$. Such is isomorphism is a representation of $C$. ${\bf RCA}_{\alpha}$ is the class of all representable cylindric algebras that have dimension $\alpha$. In particular, we are interested in the case when $\alpha = \omega$.

\begin{defin}
  Given a cylindric algebra of dimension $\alpha$ $C$, let $x$ be a term of its signature, the substitution operator $s^{i}_{j}$ have the following definition:
  \begin{center}
  $s^{i}_{j} x = \begin{cases} x, \text{if } i = j \\ c_i (d_{ij} \land x), \text{otherwise} \end{cases}$
  \end{center}
\end{defin}

It is well known that ${\bf RCA}_{\alpha}$ is a variety, ${\bf RCA}_{\alpha}$ ($\alpha \leq 2$) is finitely axiomatisable and ${\bf RCA}_{\alpha}$ ($2 < \alpha < \omega$) has no finite axiomatisation, see \cite{Henkin1988-HENCAP-4}.

Let $\mathcal{A} \in {\bf CA}_{\omega}$, then $\mathcal{A}$ has a \emph{complete representation}, if this representation preserves all existing suprema. In other words, $\mathcal{A}$ is completely representable.

Let us concretise the definition of a canonical extension for ${\bf CA}_{\alpha}$-type BAOs.

\begin{defin}
  Let $\mathcal{C} = \langle C, +, -, 0, 1, \{ d_{ij} \}_{i, j < \alpha}, \{c_i\}_{i < \alpha} \rangle$ $\mathcal{A}$ be a BAO of type ${\bf CA}_{\alpha}$ Let ${\bf Uf}(\mathcal{C})$ be the set of all ultrafilters of $\mathfrak{B}\mathcal{C}$, the Boolean part of $\mathcal{C}$.

  Let us define ${\bf C}_{i} : {\bf Uf}(\mathcal{C}) \to {\bf Uf}(\mathcal{C})$ for each $i, j < \alpha$ as
  \begin{enumerate}
\item ${\bf C}_{i} \mathcal{X} = \{ \mathcal{F} \in {\bf Uf}(\mathcal{C}) \: | \: \exists \mathcal{F}' \in {\bf Uf}(\mathcal{C}) \: (a \in \mathcal{F} \Rightarrow c_i a \in \mathcal{F}' R)\}$,
\item $D_{ij} = \{ \mathcal{F} \in {\bf Uf}(\mathcal{C}) \: | \: d_{ij} \in \mathcal{F} \}$.
  \end{enumerate}
  The structure $\mathcal{C}^{+} = \langle  {\bf Uf}(\mathcal{C}), \cup, -, \emptyset, C, {\bf C}_{i < \alpha}, \{D_{ij}\}_{i, j < \alpha} \rangle$ is called the canonical extension of $\mathcal{C}$.
\end{defin}

\subsection{Closed elements}

The following definitions and facts are due to Henkin, Monk, and Tarski \cite{henkin1971cylindric}.

Let $\mathcal{A} \in {\bf CA}_{\alpha}$ and $x \in \mathcal{A}$. Recall that \emph{the dimension of $x$}
is the set of all ordinals $\gamma < \alpha$ such that $c_\gamma x \neq x$. More formally,
\begin{center}
  $\Delta x = \{ \gamma \: | \: \gamma < \alpha \: \& \: c_\gamma x \neq x\}$
\end{center}

Let us discuss some metamathematical intuitions standing behind the notion of a dimension. Let $\Theta$ be a first-order theory and $\mathcal{C}/\equiv_{\Theta}$ its Lindenbaum-Tarski algebra. Let $\varphi$ be a formula in the signature of $\Theta$. Then $\Delta(\varphi/\Theta)$ consists of all $\kappa < \alpha$ such that $\exists x_{\kappa} \varphi \leftrightarrow \varphi$ is not valid in $\Theta$.
That is, $\Delta(\varphi/\Theta)$ contains ordinals $\kappa$ for which $x_{\kappa}$ is free in $\varphi$. Moreover, $\Delta(\varphi/\Theta)$ consists only of those ordinals for which $x_{\kappa}$ is free in every $\psi \in \varphi/\Theta$.

In particular, an element $x$ is called \emph{zero-dimensional} if $\Delta x = 0$. Zero-dimensional elements correspond to equivalence classes of sentences in the Lindenbaum-Tarski algebra of a given first-order theory. Thus, the set of zero-dimensional elements form a Boolean algebras of sentences associated with $\Theta$.

\begin{defin}
  Let $\mathcal{A}$ be an $\alpha$-dimensional cylindric algebra. Let $\alpha$ be an ordinal and $\Gamma$ a subset $\alpha$, then an element $x \in \mathcal{A}$ is $\Gamma$-closed if $\Delta x \cdot \Gamma = \emptyset$. Alternatively, $x$ is a $\Gamma$-cylinder.

 $\operatorname{Cl}_{\Gamma}\mathcal{A}$ is the set of all $\Gamma$-closed elements.
\end{defin}

Metamathematically, $\Gamma$-closed elements reflect universal closures (is it correct?).

\subsection{Neat reducts}

Let $\mathcal{C} = \langle C, +, -, 0, 1, \{d_{ij}\}_{i, j < \beta}, \{ c \}_{c < \beta} \rangle$ be a $\beta$-dimensional cylindic algebra and $\alpha \leq \beta$ an ordinal. The \emph{$\alpha$-th reduct} of $\mathcal{C}$, denoted as $\mathfrak{R}\mathfrak{d}_{\alpha}\mathcal{C}$, is an algebra having the form
\begin{center}
$\mathfrak{R}\mathfrak{d}_{\alpha}\mathcal{C} = \langle C, +, -, 0, 1, \{d_{ij}\}_{i, j < \alpha}, \{ c \}_{c < \alpha} \rangle$
\end{center}

$\mathcal{B}$ is a subreduct of $\mathcal{C}$, denoted as $\mathcal{B} \subseteq^{r} \mathcal{C}$, if $\mathcal{B} \subseteq \mathfrak{R}\mathfrak{d}_{\gamma}\mathcal{C}$ for some $\gamma \leq \beta$.

\begin{defin}
  Let $\mathcal{C}$ be a $\beta$-dimensional cylindic algebra and $\alpha$ an ordinal such that $\alpha \leq \beta$.
  The neat $\alpha$-reduct of $\mathcal{C}$, denoted as $\mathfrak{N}\mathfrak{r}_\alpha \mathcal{C}$,
  is the subalgbera $\mathcal{A}$ of $\mathfrak{R}\mathfrak{d}_{\alpha}\mathcal{C}$ with $\mathcal{A} = \operatorname{Cl}_{\kappa} \mathcal{C}$ where $\alpha + \kappa = \beta$.

  Let $\mathbb{K}$ be a class of $\beta$-dimensional cylindic algebras, then we put
  \begin{center}
    ${\bf Nr}_{\alpha}\mathbb{K} = \{ \mathfrak{N}\mathfrak{r}_\alpha \mathcal{C} \: | \: \mathcal{C} \in \mathbb{K} \}$
  \end{center}

  An algebra $\mathcal{B}$ is a neat subreduct of $\mathcal{C}$, or $\mathcal{B}$ is neatly embeddable to $\mathcal{C}$ if there exists an ordinal $\gamma \leq \alpha$ such that
  $\mathcal{C} \subseteq \mathfrak{R}\mathfrak{d}_{\gamma}\mathcal{B}$.
\end{defin}

One may define neat reducts alternatively as follows.
Let $\mathcal{C}$ be a $\beta$-dimensional cylindic algebra and $\alpha$ an ordinal such that $\alpha \leq \beta$. The neat $\alpha$-reduct of $\mathcal{C}$ is the $\alpha$-dimensional cylindric algebra having the form

\begin{center}
  $\mathfrak{N}\mathfrak{r}_\alpha \mathcal{C} = \langle \{ a \in \mathcal{C} \: | \: \forall j (\alpha \leq j \: \& \: j < \beta \Rightarrow c_j a = a) \}, +, -, 0, 1, \{d_{ij}\}_{i,j < \alpha}, \{c_{\gamma}\}_{\gamma} \rangle$
\end{center}

\section{Completely representable cylindric algebras of dimension $\omega$}

\begin{defin} \label{network}
  Let $\mathcal{A}$ be a BAO of type ${\bf CA}_{\omega}$, an $\mathcal{A}$-pre-network is a pair
  $\mathcal{N} = \langle N, l \rangle$, where $N$ is a set of nodes and $l : \prescript{\omega}{}N \to \operatorname{At}(\mathcal{A})$.

  $\mathcal{N}$ is a network, if the following conditions hold, for all $x, y \in \prescript{\omega}{}N$ and $i, j < \omega$:
  \begin{enumerate}
    \item $l(x) \leq d_{ij}$ iff $x_i = x_j$
    \item $x \equiv_i y$ implies $l(x) \leq c_i l(y)$
  \end{enumerate}
\end{defin}

Let $\mathcal{N}_1 = \langle N_1, l_1 \rangle$ and $\mathcal{N}_2 = \langle N_2, l_2 \rangle$ be networks, then $\mathcal{N}_1 \subseteq \mathcal{N}_2$ if $N_1 \subseteq N_2$ and $l_1 = l_2 	\upharpoonright_{N_1}$.

Let $\Lambda \in \operatorname{Lim}$ and $\{ \mathcal{N}_{\lambda}\}_{\lambda < \Lambda}$ a sequence of networks such that
\begin{center}
  $\langle N_0, l_0 \rangle \subseteq \langle N_1, l_1 \rangle \subseteq \dots \langle N_{\lambda}, l_{\lambda} \rangle \subseteq \dots $ for $\lambda < \Lambda$
\end{center}
then \emph{the limit} of the sequence $\{ \mathcal{N}_{\lambda}\}_{\lambda < \Lambda}$ is the network
\begin{center}
  $\mathcal{N} = \langle N, l \rangle = \bigcup \limits_{\lambda < \Lambda} \langle N_{\lambda}, l_{\lambda} \rangle$
\end{center}
with nodes $N = \bigcup \limits_{\lambda < \Lambda} N_{\lambda}$ and labelling $l = \bigcup \limits_{\lambda < \Lambda} l_{\lambda}$, that is,
  for any $\lambda \in \Lambda$ and $x \in \prescript{\omega}{}N$ one has $l(x) = l_{\lambda}(x)$.

The elements of $\prescript{\omega}{}N$ are called \emph{$\omega$-dimensional hyperedges} of a network. One may identify a complete representation of an atomic cylindric-type algebra $\mathcal{A}$
with a set $\{ \mathcal{N}_a \: | \: a \in \operatorname{At}(\mathcal{A}) \}$ of $\mathcal{A}$-networks with the following additional condition:
\begin{itemize}
  \item For each $a \in \operatorname{At}(\mathcal{A})$ there exists $x \in \prescript{\omega}{}N_a$ such that $l_a(x) = a$ and for each $z \in \prescript{\omega}{}N_a$ and $b \in \operatorname{At}(\mathcal{A})$, $i < \omega$ with $l_a(z) \leq c_i b$
  there exists $y \in \prescript{\omega}{}N_a$ such that $z \equiv_i y$ and $l_a(y) = b$.
\end{itemize}

We define a complete representation $h$ of a cylindric-type algebra $\mathcal{A}$ as follows, for any $b \in \mathcal{A}$:
\begin{center}
$h(b) = \{ x \: | \: \exists a \in \operatorname{At}(\mathcal{A}), x \in \prescript{\omega}{}N_a, l_a(x) \leq b \}$
\end{center}

Let us define an atomic game.

\begin{defin} Let $n \leq \omega$ an $\mathcal{A}$ be an atomic BAO of type ${\bf CA}_{\omega}$ that has countably atoms. The game $\mathcal{G}_{n}(\mathcal{A})$ is defined as follows. The game has two players: $\forall$ (Abelard, he/his) and $\exists$ (H\'{e}lo\"{i}se, she/her). A play of the game $\mathcal{G}_{n}(\mathcal{A})$ is the sequence of networks
  \begin{center}
    $\mathcal{N}_0 \subseteq \mathcal{N}_1 \subseteq \mathcal{N}_2 \subseteq \dots \subseteq \mathcal{N}_{k} \subseteq \dots$ for $k < n$
  \end{center}
  The game consists of the following stages:
  \begin{enumerate}
    \item ({\bf Zero round})

    $\forall$ picks an atom $a \in \operatorname{At}(\mathcal{A})$ and $\exists$ plays a network $\mathcal{N}_0$. If there is no $x \in \prescript{\omega}{}N_0$ such that $l_0(x) = a$, then $\forall$ wins the play.
    \item ({\bf Successor round})

    Let $k < n$ such that $k + 1 < \kappa$ and a network $\mathcal{N}_{\lambda} = \langle N_{k}, l_{k} \rangle$ has been already played.

    $\forall$ picks $i \leq k$, $x \in \prescript{\omega}{}N_{k}$, $a \in \operatorname{At}(\mathcal{A})$ such that $l_{k}(x) \leq c_i a$.
    We denote this move as $(i, x, a)$. $\exists$ responds with a network $\mathcal{N}_{k + 1} \supseteq \mathcal{N}_{k}$. $\forall$ wins, if there is no node $c \in N$ such that $l_{\lambda+1}(x[i/c]) = a$, then $\forall$ wins
    \item The limit of the play is defined as $\bigcup \limits_{k < n} \mathcal{N}_{k}$. $\forall$ wins the play, if there exists $m < n$ such that $\exists$ does not win the $m$-th round. Otherwise, $\exists$ wins the play.
  \end{enumerate}
\end{defin}

\begin{theorem}
  Let $\mathcal{A}$ be an atomic $\omega$-dimensional cylindric-type algebra and $\kappa$ a cardinal such that $|\operatorname{At}(\mathcal{A})| = \kappa$, then the following are equivalent:

  \begin{enumerate}
    \item $\mathcal{A}$ is completely representable.
    \item $\exists$ has a winning strategy in $\mathcal{G}_{\kappa + \omega}$.
  \end{enumerate}
\end{theorem}

\begin{proof}
  $ $

  \begin{enumerate}
    \item $\Rightarrow$
    If $\mathcal{A}$ is completely representable, then its Boolean reduct is completely representable as well by Theorem~\ref{completeboolean}. $\exists$ maintains that embedding to win the play. TODO: write down this proof in more detail

    \item $\Leftarrow$

    Suppose $\exists$ has a winning strategy in $\mathcal{G}_{\kappa + \omega}(\mathcal{A})$.
    In every round $\forall$ picks all possible $i < \omega$, $a \in \operatorname{At}(\mathcal{A})$, all possible hyperedges and all appropriate atoms and $\exists$ has a proper response for every $\forall$'s move.

    For each atom consider a play of the game with fewer than $\kappa + \omega$ nodes. For each $a \in \operatorname{At}(\mathcal{A})$ we associate a network $\mathcal{N}_a$, the resulting network of a corresponding game. Consider the set $\{ \mathcal{N}_a \: | \: a \in \operatorname{At}(\mathcal{A}) \}$.

    Let $a$ be an atom, consider the network $\mathcal{N}_a = \langle V, l_a \rangle$. If there was not $x \in V$ such that $l_a(x) = a$, then $\forall$ would have a winning strategy, but that is not true, such an $x$ does exist. The second item of this criterion follows from the presence of a winning strategy for $\exists$ as well.

    So we define a map $rep$:
    \begin{center}
    $\operatorname{rep}(a) = \{ x \: | \: \exists b \in \operatorname{At}(\mathcal{A}) \: x \in \prescript{\omega}{}N_a, l_a(x) \leq b \}$.
  \end{center}
    We check that $rep$ preserves cylindrifications and diagonal elements. Let $i, j < \omega$ and $a \in \mathcal{A}$:
    \begin{enumerate}
    \item Suppose $x \in \operatorname{rep}(c_i a)$, then there exists an atom $b$ such that $x \in \prescript{\omega}{}N_b$ with $l_b(x) \leq c_i a$. Then there exists $y \equiv_i x$ with $l_b(y) \leq a$, so $x \in {\bf C}_i(\operatorname{rep}(a))$.

    Let $x \in {\bf C}_i(\operatorname{rep}(a))$. We need $x \in (\operatorname{rep}(c_i a))$, that is, one needs to find an atom $c$ such that $l_c(x) \leq c_i a$.

    We already know that there exists $y \equiv_i x$ such that $y \in \operatorname{rep}(a)$, that is, there exists an atom $b$ such that $y \in \prescript{\omega}{}N_b$ and $l_b(y) \leq a$.

    \item If $x \in \operatorname{rep}(d_{ij})$, so there exists an atom $b$ with $x \in \prescript{\omega}{}N_a$ and $l_b(x) \leq d_{ij}$, then $x_i = x_j$, then $x \in D_{ij}$.
    \end{enumerate}
  \end{enumerate}
\end{proof}

If $\mathcal{A} \in {\bf CA}_{n}$ is atomic, then we have the following criterion of complete representability formulated in terms of back-and-forth games, see \cite[Theorem 3.3.3 and Corollary 3.3.5]{hirsch2013completions}. Here, the definition of a network for the case of ${\bf CA}_n$ is the same as Definition~\ref{network}, but hyperedges have lenght $n$.

\begin{theorem}
  Let $\mathcal{A}$ be an atomic BAO of type ${\bf CA}_n$ such that $|\operatorname{At}(\mathcal{A})| = \kappa$, where $\kappa$ is an arbitrary cardinal. Then the following are equivalent:

  \begin{enumerate}
    \item $\mathcal{A}$ is completely representable iff $\exists$ has a winning strategy in $\mathcal{G}_{\kappa + \omega}(\mathcal{A})$.
    \item $\exists$ has a winning strategy in $\mathcal{G}_{m}(\mathcal{A})$ for every $m < \omega$ iff $\mathcal{A}$ is elementarily equivalent to a completely representable cylindric algebra.
  \end{enumerate}
\end{theorem}

Now we formulate the Lyndon conditions for BAOs of type ${\bf CA}_{\omega}$, but we need their analogues for the finite case which are due to \cite[Theorem 34]{hirsch1997step}. Let $\mathcal{A}$ be an atomic BAO of type ${\bf CA}_{n}$. $lc_k^n$ is a first-order sentence such that $\mathcal{A} \models lc_k^n$ iff $\exists$ has a winning strategy in the game $\mathcal{G}_k(\mathcal{A})$ of length $k < \omega$.

Consider the following set of formulas:
\begin{center}
  $LC_{\omega} = \{ lc_n^n \: | \: n < \omega \}$
\end{center}
We shall think of $LC_{\omega}$ as the Lyndon conditions for atomic cylindric algebras of dimension $\omega$. Let us write down those $lc_n^n$s explicitly, this is due to \cite[Sections 11.3 and 11.4]{hirsch1997step}. For that, we need a bit of definitions.

\begin{proposition}
  Let $\mathcal{A}$ be a BAO of type ${\bf CA}_{\omega}$ and $m < \omega$, then $\exists$ has a winning strategy in $\mathcal{G}_m(\mathcal{A})$ iff $\mathcal{A}_m \models lc(m,m)$
\end{proposition}

\begin{proof}
\end{proof}

Let us recall the definition of an ultraproduct, see \cite[Section 9.5]{hodges1993model}. Let $I$ be an index set, $U$ an ultrafiler over $I$, an $(\mathcal{A}_i)_{i \in I}$ a family of structures of the same signature. The ultraproduct of $(\mathcal{A}_i)_{i \in I}$ modulo $U$, denoted as $\Pi_U \mathcal{A}_i$, is an algebra of the same signature whose elements are elements of the direct product
$\Pi_{i \in I} \mathcal{A}_i$ factorised through an equivalence relation $\sim$ such that for every $a, b \in \Pi_{i \in I} \mathcal{A}_i$:
\begin{center}
  $a \sim b \Leftrightarrow \{ i \in I \: | \: a(i) = b(i)\} \in U$
\end{center}

\begin{lemma}
  Let $\mathcal{A}$ be an atomic BAO of type ${\bf CA}_{\omega}$ such that for every $n < \omega$ $\mathcal{A}_n \models lc(n,n)$, then $\exists$ has a winning strategy in $\mathcal{G}_{\omega}(\Pi_U \mathcal{A})$, where $\Pi_U \mathcal{A}$ is a non-principal ultrapower of $\mathcal{A}$ and $U$ is a non-principal ultrafilter over $\omega$.
\end{lemma}

\begin{proof}
  Suppose $\mathcal{A}_n \models lc(n,n)$ for each $n$.
\end{proof}

\begin{theorem}
  What's going on? :(
\end{theorem}

\begin{theorem} Let $\mathcal{A}$ be a BAO of type ${\bf CA}_{\omega}$:

\begin{enumerate}
  \item $\exists$ has a winning strategy in $\mathcal{G}_{m}(\mathcal{A})$ ($m < \omega$), then $\exists$ has a winning strategy in $\mathcal{G}_{\omega}(\Pi_U \mathcal{A})$, where $\Pi_U \mathcal{A}$ is the non-principal ultrapower of $\mathcal{A}$ modulo $U$, an ultrafilter over $\omega$.
  \item $\exists$ has a winning strategy in $\mathcal{G}_{m}(\mathcal{A})$ (for every $m < \omega$) iff $\mathcal{A}$ is elementarily equivalent to a completely representable cylindric algebra of dimension $\omega$.
\end{enumerate}
\end{theorem}

\begin{proof}
  $ $

  The argument uses \L o\'s's Theorem, see \cite[Theorem 9.5.1]{hodges1993model}.

  \begin{enumerate}
    \item
    \item
  \end{enumerate}
\end{proof}

\begin{theorem}
  The elementary closure of the class of completely representable cylindric algebras of dimension $\omega$ is axiomatised with the axioms of ${\bf CA}_{\omega}$ and the Lyndon conditions.
\end{theorem}

\section{The result itself}

\begin{lemma}\label{Neat}
  Let $\mathcal{A}$ be a BAO of type ${\bf CA}_{\alpha}$ and $\mathcal{B}$ be a $\beta$-dimensional cylindric algebra such that $\beta \leq \alpha$ and $\mathcal{A}$ neatly embeds to $\mathcal{B}$ by a complete embedding.

  \begin{enumerate}
  \item $\mathcal{A}^{+}$ neatly embeds to $\mathcal{B}^{+}$ by a complete embedding.
  \item $\mathcal{A}$ is atomic.
  \end{enumerate}
\end{lemma}

\begin{proof}
$ $

\begin{enumerate}
  \item See \cite[Remark 2.7.25]{henkin1971cylindric}.
  \item Is it true?
\end{enumerate}
\end{proof}

\begin{theorem}[This assumption is by Ian Hodkinson]\label{Neat2}
  $ $

  Let $\mathcal{A}$ be a BAO of type ${\bf CA}_{\omega}$ such that $\mathcal{A}$ neatly embeds into ${\bf CA}_{\omega + \omega}$ by a complete embedding. Then $\mathcal{A}$ is completely representable as ${\bf CA}_{\omega}$.
\end{theorem}

\begin{proof}
  Suppose $\mathcal{A} \subseteq \mathfrak{Nr}_{\omega} \mathcal{B}$, where $\mathcal{B} \in {\bf RCA}_{\omega + \omega}$ and the inclusion map $\rho : \mathcal{A} \hookrightarrow \mathfrak{Nr}_{\omega} \mathcal{B}$ is a complete embedding, that is:
  \begin{center}
    $\rho (\sum \limits_{i \in I} a_i) = \sum \limits_{i \in I} (\rho a_i)$, if $\sum \limits_{i \in I} a_i$ exists.
  \end{center}

  Let us show that $\mathcal{A}$ is atomic.

  Consider $\rho(\mathcal{A})$. Let us show that $\exists$ has a winning strategy on $\mathcal{G}_{\kappa + \omega}(\rho(\mathcal{A}))$
\end{proof}

Lemma~\ref{Neat} and Theorem~\ref{Neat2} imply the following theorem.

\begin{theorem}
  Let $\mathcal{C} \in {\bf RCA}_{\omega}$, then $\mathcal{C}^{+} \in {\bf RCA}_{\omega}$. That is, ${\bf RCA}_{\omega}$ is closed under canonical extensions.
\end{theorem}

\begin{proof}

\end{proof}

\section{(Lack of) canonical axiomatisation of ${\bf CA}_{\omega}$}

Here we are going to show that ${\bf CA}_{\omega}$ fails to have a canonical axiomatisation, the similar results for ${\bf RRA}$ and ${\bf RCA}_n$ for finite $n \geq 3$ have been shown by Hodkinson and Venema \cite{hodkinson2005canonical} and by Bulian and Hodkinson respectively \cite{bulian2013bare}.

\section{Notes on the canonicity of ${\bf RRA}$}

\begin{defin}
  $ $

    A relation algebra is an algebra $\mathcal{R} = \langle R, 0, 1, +, -, ;, {}^{\smile}, {\bf 1 }' \rangle$ such that $\langle R, 0, 1, +, - \rangle$ is a Boolean algebra and the following
    equations hold, for each $a, b, c \in R$:
    \begin{enumerate}
      \item $a ; (b ; c) = (a ; b) ; c$
      \item $(a + b) ; c = (a ; c) + (b ; c)$
      \item $a ; {\bf 1}' = a$
      \item $a^{\smile \smile} = a$
      \item $(a + b)^{\smile} = a^{\smile} + b^{\smile}$
      \item $(a ; b)^{\smile} = b^{\smile} ; a^{\smile}$
      \item $a^{\smile} ; (- (a ; b)) \leq - b$
    \end{enumerate}
    where $a \leq b$ iff $a + b = b$. ${\bf RA}$ denotes the class of all relation
    algebras.
\end{defin}

We will adapting the following proof of the fact that ${\bf RRA}$ is canonical \footnote{This idea is by Ian Hodkinson} to our case. This proof is due to Monk, but that was describe in McKenzie's thesis \cite{mckenzie1968representation}.

\begin{enumerate}
  \item A relation algebra $\mathcal{A}$ is representable iff $\mathcal{A}$ neatly embeds to some $\omega$-dimensional cylinric algebra,
  \item If $\mathcal{A}$ neatly embeds in $\mathcal{A}$ then $\mathcal{A}^{+}$ neatly embeds in $\mathcal{B}^{+}$,
  \item ${\bf CA}_{\alpha}$ is closed under canonical extensions,
  \item Voil\'{a}.
\end{enumerate}

\begin{defin}
  Let $\mathcal{C} \in {\bf CA}_{\alpha}$, where $\alpha \geq 3$. The relation algebra reduct of $\mathcal{C}$, written as $\mathfrak{Ra}(\mathcal{C})$, is the algebra having the form

  \begin{center}
    $\langle \operatorname{dom}(\mathfrak{Nr}_2(\mathcal{C})), 0, 1, +, -, {\bf 1}', \smile, ; \rangle$
  \end{center}
  where:

  \begin{enumerate}
    \item $+$, $-$, $0$, and $1$ are defined as usual in $\mathcal{C}$,
    \item ${\bf 1}' = d_{01} \in \mathfrak{Nr}_2(\mathcal{C})$,
    \item $r^{\smile} = s^{2}_{0} s^{0}_{1} s^{1}_{2} r$ for $r \in \mathfrak{Nr}_2(\mathcal{C})$,
    \item Let $r, s \in \mathfrak{Nr}_2(\mathcal{C})$, then $r ; s = c_2 (s^{1}_{2}r \cdot s^{0}_{2}s)$
  \end{enumerate}
\end{defin}

Moreover, $\mathfrak{Nr}_{\beta}(\mathcal{C})$ and $\mathfrak{Ra}(\mathcal{C})$ are closed under these operations.
There is also the following fact by due to Henkin, Monk, and Tarski \cite{Henkin1988-HENCAP-4}:
\begin{theorem}
  Let $\mathcal{C} \in {\bf CA}_{\alpha}$ for $\alpha \geq 4$, then $\mathfrak{Ra}(\mathcal{C})$ is a relation algebra.
\end{theorem}

The following characterisation results are by Henkin, Monk, and Tarski \cite[5.3.13, 5.3.16]{Henkin1988-HENCAP-4} as well:
\begin{theorem} \label{char}
  $ $

  \begin{enumerate}
    \item ${\bf RA} = {\bf S} \mathfrak{Ra} {\bf CA}_4$,
    \item ${\bf RRA} = \bigcap \limits_{3 \leq n < \omega} {\bf S} \mathfrak{Ra} {\bf CA}_n =  {\bf S} \mathfrak{Ra} {\bf CA}_{\alpha}$, where $\alpha$ is an infinite ordinal.
  \end{enumerate}
\end{theorem}


Let $\mathcal{C} \in \mathcal{CA}_{\alpha}$, then $\mathcal{R} \in {\bf RA}$ neatly embeds to $\mathcal{C}$, if $\mathcal{R}$ is isomorphic to some subalgebra of $\mathfrak{Ra}(\mathcal{C})$.

\begin{theorem}
  ${\bf RRA}$ is closed under canonical extensions.
\end{theorem}

\begin{proof} Let ${\bf R} \in {\bf RRA}$.
  By the second item of~\ref{char}, every representable relation algebra is isomorphic to some subalgbera of the relation algebra reduct $\mathfrak{Ra}\mathcal{C}$ for some $\mathcal{C} \in {\bf CA}_{\omega}$. But neat embeddings repsect canonical extensions, so if ${\bf R} \hookrightarrow_{n} \mathcal{C}$, so is ${\bf R}^{+} \hookrightarrow_{n} \mathcal{C}^{+}$.
  ${\bf CA}_{\alpha}$ is closed under canonical extensions, so is ${\bf RRA}$.
\end{proof}


\bibliographystyle{plain}
\bibliography{Text}

\end{document}

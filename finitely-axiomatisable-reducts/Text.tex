\documentclass[a4paper]{article}
\usepackage{amsmath}
\usepackage{amsthm}
\usepackage{amsfonts}
\usepackage{amssymb}
\usepackage{bussproofs}
\usepackage{mathtools}
\usepackage{verbatim}
\usepackage{dsfont}
\usepackage{mathabx}
\usepackage[all, 2cell]{xy}
\usepackage[all]{xy}
\usepackage{wasysym}
\usepackage{rotating}
\usepackage{geometry}
\usepackage{trfsigns}
\usepackage{cmll}
\usepackage{authblk}
\usepackage{hyperref}
\usepackage{cleveref}
\usepackage{lipsum}
\usepackage{extpfeil}
\usepackage{soul}
\usepackage{graphicx}

\newcommand\mapsfrom{\mathrel{\reflectbox{\ensuremath{\mapsto}}}}

\theoremstyle{defin}
\newtheorem{definition}{Definition}

\theoremstyle{theorem}
\newtheorem{theorem}{Theorem}

\theoremstyle{claim}
\newtheorem{claim}{Claim}

\theoremstyle{prop}
\newtheorem{prop}{Proposition}

\theoremstyle{lemma}
\newtheorem{lemma}{Lemma}

\theoremstyle{fact}
\newtheorem{fact}{Fact}

\theoremstyle{ex}
\newtheorem{ex}{Example}


\theoremstyle{col}
\newtheorem{col}{Corollary}

\let\strokeL\L
\DeclareRobustCommand{\L}{\ifmmode\mathbf{L}\else\strokeL\fi}

\author{Daniel Rogozin}
\date{}
\title{Finitely axiomatisable varieties generated by representable relation algebra reducts}

\begin{document}
\maketitle

\nocite{*}

\section{Questions}

\begin{enumerate}
\item Is the variety ${\bf V}({\bf R}(\cdot, ;, {\bf 1}))$ finitely axiomatisable?
\item Is the variety ${\bf V}({\bf R}(\cdot, +, ;, {\bf 1}))$ finitely axiomatisable?
\end{enumerate}

\section{Definitions}

A structure $\mathcal{M} = (M, \cdot, ;, {\bf 1})$ is called a \emph{lower-semilattice ordered monoid} if the following axioms hold:
\begin{itemize}
\item $(M, \cdot)$ is a meet-semilattice,
\item $(M, ;, {\bf 1})$ is a monoid,
\item $({\bf 1} \cdot x) ; ({\bf 1} \cdot y) = {\bf 1} \cdot x \cdot y$,
\item $({\bf 1} \cdot x) ; (y \cdot z) = ({\bf 1} \cdot x) ; y \cdot z$,
\item $(x \cdot y) ; ({\bf 1} \cdot z) = x \cdot y ; ({\bf 1} \cdot z)$.
\end{itemize}
${\bf LSMod}$ is the class (variety) of all lower-semilattice ordered monoids.

A lower-semilattice ordered monoid $\mathcal{M}$ is \emph{representable} if there exists a map $f : \mathcal{M} \to 2^{X}$ for some $X \neq \emptyset$ such that:
\begin{itemize}
\item $f$ is one-to-one
\item $f(a ; b) = f(a) | f(b)$
\item $f({\bf 1}) = {\bf Id}$
\item $f(a \cdot b) = f(a) \cap f(b)$
\end{itemize}
${\bf R}(\cdot, ;, {\bf 1})$ is the class of all representable lower-semilattice ordered monoids

\subsection{Stone-style representation for lower-semilattices}

Let $\mathcal{L}$ be a lower semilattice.

A subset $F \subseteq \mathcal{F}(\omega)$ is a filter if $F$ is a downward closed meet-subsemilattice. $F$ is principal if there is some $a \in F$ such that $F = \uparrow a$. Let $A \subseteq \mathcal{F}(\omega)$, then $\langle A \rangle = \cup_{a \in A} \uparrow a$, the filter generated by $F$.

A subset $I \subseteq \mathcal{L}$ is an \emph{ideal} if it is upward closed and updirected, that is, $a, b \in I$ implies there exists some $c$ such that $a, b \leq c$. It is known that $F$ is a prime filter iff $L \setminus F$ is a prime ideal.

\begin{theorem}
Let $\mathcal{L}$ be a lower semilattice. Then $\mathcal{L} \hookrightarrow (2^{\operatorname{Spec}(\mathcal{L})})$ whenever $\mathcal{L}$ is distributive.
\end{theorem}

\subsection{Finite axiomatisability}

\begin{theorem}
${\bf LSMod} = {\bf V}({\bf R}(\cdot, ;, {\bf 1}))$.
\end{theorem}

The right-to-left inclusion is obvious. To show the left-to-right inclusion, one needs to show that the free lower-semilattice ordered monoid with $\omega$ generators $\mathcal{F}(\omega)$ is representable.

A network is a structure $\mathcal{N} = (V, E, l)$, where $V$ is a set of vertices, $E$ is a set of edges and $l : E \to 2^{\mathcal{F}(\omega)}$ is a labelling function with the following data:
\begin{itemize}
\item Each $l(x,y)$ is a filter,
\item $l(x,y) ; l(y, z) \subseteq l(y, z)$,
\item ${\bf 1'} \in l(x, y)$ implies $x = y$,
\item
\end{itemize}

We define a back-and-forth representability game $\mathcal{G}_{\omega}(\mathcal{F}(\omega))$ with two players $\forall$ and $\exists$.

\section{The distributive lattice case}

\begin{theorem}
Let $\mathcal{L}$ be a distributive lattice, then $\mathcal{L} \hookrightarrow 2^{\operatorname{Spec}(\mathcal{L})}$.
\end{theorem}


Given a distributive-lattice ordered monoid $\mathcal{M}$, its canonical extension is a structure $\mathcal{M}_{+} = (\operatorname{Spec}(\mathcal{M}), \subseteq, R, E)$, where
\begin{itemize}
\item
\item
\item
\end{itemize}

\begin{definition}
Join-irreducibles...
\end{definition}

TODO: complete representation vs atomic representation vs representations via join-irreducibles

TODO: Birkhoff representation

TODO: Raney representation

\begin{theorem}
$((\mathcal{F}_{\omega})_+)^+$ is completely representable.
\end{theorem}

\bibliographystyle{alpha}
\bibliography{Text}

\end{document}

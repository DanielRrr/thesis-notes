\documentclass[a4paper]{article}
\usepackage{amsmath}
\usepackage{amsthm}
\usepackage{amsfonts}
\usepackage{amssymb}
\usepackage{bussproofs}
\usepackage{mathtools}
\usepackage{verbatim}
\usepackage{dsfont}
\usepackage{mathabx}
\usepackage[all, 2cell]{xy}
\usepackage[all]{xy}
\usepackage{wasysym}
\usepackage{rotating}
\usepackage{geometry}
\usepackage{trfsigns}
\usepackage{cmll}
\usepackage{authblk}
\usepackage{hyperref}
\usepackage{cleveref}
\usepackage{lipsum}
\usepackage{extpfeil}
\usepackage{soul}
\usepackage{graphicx}

\newcommand\mapsfrom{\mathrel{\reflectbox{\ensuremath{\mapsto}}}}

\theoremstyle{defin}
\newtheorem{defin}{Definition}

\theoremstyle{theorem}
\newtheorem{theorem}{Theorem}

\theoremstyle{prop}
\newtheorem{prop}{Proposition}

\theoremstyle{lemma}
\newtheorem{lemma}{Lemma}

\theoremstyle{ex}
\newtheorem{ex}{Example}


\theoremstyle{col}
\newtheorem{col}{Corollary}

\author{Daniel Rogozin}
\date{}
\title{Stuff related to tensor products of modal algebras}

\begin{document}

\maketitle

\nocite{*}

\section{Preliminaries}

The basic definitions according to \cite{chagrov1997modal}.

\begin{defin} An $n$-modal algebra is an algebra $\mathcal{B} = \langle B, \vee, \neg, \bot, \Diamond_1, \dots, \Diamond_n \rangle$, where $\mathcal{B} = \langle B, \vee, \neg, \bot \rangle$ is a Boolean algebra and for each $i = 1,\dots,n$ $\Diamond_i$ is an order-preserving unary function obeying:
  \begin{enumerate}
    \item $\Diamond_i p \vee \Diamond_i q = \Diamond_i (p \vee q)$
    \item $\Diamond_i \bot = \bot$
  \end{enumerate}
\end{defin}

\begin{defin} An $n$-normal modal logic is a set of formulas containing Boolean tautologies, formulas $\Diamond_i p \vee \Diamond_i q \leftrightarrow \Diamond_i (p \vee q)$ and $\Diamond_i \bot \leftrightarrow \bot$ for $i = 1,\dots,n$, and is closed under ${\bf MP}$, ${\bf Sub}$, and the monotonicity rule: from $\varphi \to \psi$ infer $\Diamond_i \varphi \to \Diamond_i \psi$.
\end{defin}

\begin{defin}
$ $
\begin{enumerate}
  \item A Kripke $n$-frame is a structure $\mathcal{F} = \langle W, R_1, \dots, R_n \rangle$, where $R_i \subseteq W \times W$ for each $i = 1,\dots, n$.
  \item Let $\mathcal{F}$ be a Kripke frame. A Kripke model over $\mathcal{F}$ is a tuple $\mathcal{M} = \langle \mathcal{F}, \vartheta \rangle$, where $\vartheta : \operatorname{PV} \to 2^W$ is a valuation. Modalised formulas have the following semantics:
  \begin{center}
    $\mathcal{M}, x \models \Diamond_i \varphi \Leftrightarrow \exists y \in R_i(x) \: \mathcal{M}, y \models \varphi$.
  \end{center}
  A formula $\varphi$ is true in $\mathcal{M}$ iff $||\varphi||_{\mathcal{M}} = W$. $\varphi$ is valid in $\mathcal{F} = \langle W, R_1, \dots, R_n \rangle$ iff $||\varphi||_{\mathcal{M}} = W$ for every model $\mathcal{M}$ over $\mathcal{F}$.
  \item The complex algebra of a Kripke frame $\mathcal{F} = \langle W, R_1, \dots, R_n \rangle$ is a complete modal algebra defined as $\mathcal{F}^{+} = \langle \mathcal{P}(W), \cup, \neg, \emptyset, R_1^{-1}, \dots, R_n^{-1} \rangle$.
  \item A general $n$-frame is a structure $\mathcal{F} = \langle W, R_1, \dots, R_n, \mathcal{A} \rangle$, where $\mathcal{A}$ is a subalgebra of $\langle W, R_1, \dots, R_n \rangle^{+}$.
\end{enumerate}
\end{defin}

\begin{defin}
$ $
  Let $\mathcal{F}$ be an $n$-frame, then $\operatorname{Log}(\mathcal{F}) = \{ \varphi \: | \: \mathcal{F} \models \varphi \}$. If $\mathbb{F}$ is a class of $n$-frames,
  then $\operatorname{Log}(\mathbb{F}) = \bigcap \limits_{\mathcal{F} \in \mathbb{F}} \operatorname{Log}(\mathcal{F})$. We use the same notation for logics of general frames and modal algebras.
\end{defin}

We discuss the background on products of modal logics and products of Kripke frames \cite{kurucz2007combining} \cite{kurucz2003many}.

\begin{defin} Let $\mathcal{F}_1 = \langle W_1, R^1_1, \dots, R^1_n\rangle$ be an $n$-frame and $\mathcal{F}_2 = \langle W_2, R^2_1, \dots, R^2_m\rangle$ an $m$-frame
\end{defin}

\begin{defin} discrete duality between perfect modal algebras and Kripke frames
\end{defin}

\begin{defin} A finitely axiomatisable normal modal logic
\end{defin}

TODO: describe related underlying results

\section{Tensor products of modal algebras. The basic definitions and results}

\begin{defin} A commutative associative algebra
\end{defin}

\begin{defin} Tensor product of them
\end{defin}

\begin{defin} Tensor product of modal algebras
\end{defin}

\begin{defin} Tensor product of general frames
\end{defin}

TODO: describe related underlying results

\section{Note on incomplete modal logics}

TODO: consider the system containing ${\bf GL}$, McKinsey, seriality, and linearity.

\section{Solution for Problem 1}

\section{Solution for Problem 4}

\newpage

\bibliographystyle{plain}
\bibliography{Text}

\end{document}

\documentclass[a4paper]{article}
\usepackage{amsmath}
\usepackage{amsthm}
\usepackage{amsfonts}
\usepackage{amssymb}
\usepackage{bussproofs}
\usepackage{mathtools}
\usepackage{verbatim}
\usepackage{dsfont}
\usepackage{mathabx}
\usepackage[all, 2cell]{xy}
\usepackage[all]{xy}
\usepackage{wasysym}
\usepackage{rotating}
\usepackage{geometry}
\usepackage{trfsigns}
\usepackage{cmll}
\usepackage{authblk}
\usepackage{hyperref}
\usepackage{cleveref}
\usepackage{lipsum}
\usepackage{extpfeil}
\usepackage{soul}
\usepackage{graphicx}

\newcommand\mapsfrom{\mathrel{\reflectbox{\ensuremath{\mapsto}}}}

\theoremstyle{defin}
\newtheorem{defin}{Definition}

\theoremstyle{theorem}
\newtheorem{theorem}{Theorem}

\theoremstyle{prop}
\newtheorem{prop}{Proposition}

\theoremstyle{lemma}
\newtheorem{lemma}{Lemma}

\theoremstyle{ex}
\newtheorem{ex}{Example}


\theoremstyle{col}
\newtheorem{col}{Corollary}

\author{Daniel Rogozin}
\date{}
\title{Stuff related to tensor products of modal algebras}

\begin{document}

\maketitle

\nocite{*}

\section{Preliminaries}

\begin{defin} A normal modal logic
\end{defin}

\begin{defin} A Kripke frame, a Kripke model, a general frame
\end{defin}

\begin{defin} A modal algebra/a complex algebra
\end{defin}

\begin{defin} the logic of a frame/class of frames.
\end{defin}

\begin{defin} product of frames/product of logics
\end{defin}

\begin{defin} discrete duality between perfect modal algebras and Kripke frames
\end{defin}

\begin{defin} A finitely axiomatisable normal modal logic
\end{defin}

TODO: descibe related underlying results

\section{Tensor products of modal algebras. The basic definitions and results}

\begin{defin} A commutative associative algebra
\end{defin}

\begin{defin} Tensor product of them
\end{defin}

\begin{defin} Tensor product of modal algebras
\end{defin}

\begin{defin} Tensor product of general frames
\end{defin}

TODO: descibe related underlying results

\section{Note on incomplete modal logics}

TODO: consider the system containing ${\bf GL}$, McKinsey, seriality, and linearity.

\section{Solution of Problem 1}

\section{Solution of Problem 4}

\bibliographystyle{plain}
\bibliography{Text}

\end{document}
